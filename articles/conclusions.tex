\section{Conclusion}

We have studied the (co-)algebraic properties of monadic streams ({\em monsters}).
Specifically, we discovered that they inherit from their base operator $M$ the property of being a functor and an applicative functor, but not necessarily that of being a monad.

In the introductory motivation, we illustrated the use of monsters with several instances: for specific choices of $M$, we obtain useful data structures, like lazy lists, finitely branching trees, interactive processes, state machines.

Concurrently with this theoretical study, we also developed an extensive Haskell library for monsters.
In addition to the concrete realization of the type class instances described here, we also programmed a large collection of functions and operators on monsters.
Some of them are generalizations of analogous definitions for pure lists and streams, others are new and original.
The complete Haskell library is publicly available at this address: \repourl.

In the near future we will publish two additional articles:
first, a comprehensive description of the monster library, expounding the practical programming side of monsters, together with several applications;
second, an extension of our categorical study of type classes of monsters, with the specific goal to prove that it is impossible to give a universal monad instance for monsters.
