\section{Conclusion}

We have studied the (co-)algebraic properties of monadic streams ({\em monsters}).
Specifically, we discovered that they inherit from their base operator $M$ the property of being a functor and an applicative functor, but not that of been a monad.

In the process we developed convenient notations and proved several useful lemmas.

In the introductory motivation, we illustrated the use of monsters with several instances: for specific choices of $M$, we obtain useful data structures, like lazy lists, finitely branching trees, interactive processes, state machines.

Concurrently with this theoretical study, we also developed an extensive Haskell library for monsters.
Besides the concrete realization of the type class instances described here, we also programmed a large collection of functions and operators on monsters, some of them are generalizations of analogous definitions for pure lists and streams, others are new and original.

We will release the library soon, together with the publication of a second article expounding the practical programming side of monsters, together with several applications.
