\section{Monsters are Functors}\label{sec:functor}

In this section we assume that $M$ is a functor.
We want to show that $\stream{M}$ is also a functor: we have to define its extension to morphisms and prove the functor laws.
If $f:A\rightarrow B$, then define how $f$ maps on $M$-monsters:
$$
\begin{array}{l}
\stream{M} f : \stream{M} A \rightarrow \stream{M} B\\
\stream{M} f\,(\mcons\,m) = \mcons\,(M\,(f\times \stream{M} f)\,m)
\end{array}
$$

This definition complies with the {\em guarded-by-constructors} discipline: the recursive call to $(\stream{M} f)$ is mapped to the recursive substreams by the functorial application of the functor $M\,(A \times -)$.
That is: $(\stream{M} f)$ will be recursively applied only at the recursive positions inside the shape of $m$.
There are also applications of $f$ to the first element (of type $A$) of the pairs in the $M$-position: this is non-recursive, and therefore not problematic.


We can now prove that the functor laws are satisfied by $\stream{M}$: its functorial mapping preserves identity and composition.
The proofs are straightforward applications of definitions and the functoriality of $M$ and $\times$, except for the use of coinduction;
we are allowed to invoke the laws themselves in their proofs, as long as we use them only in the direct recursive subterms of the $\mcons$ constructor.

\begin{lemma}\label{lemma:functor_id}
The identity functor law holds for monadic streams:

$$
\stream{M} \id_A = \id_{(\stream{M} A)}
$$

\end{lemma}
\begin{proof}
We prove the statement pointwise, by applying the left-hand side function to an $M$-monster in constructor form $(\mcons\,m)$.
We simplify according the the definition of the definition of $\stream{M}$ mapping:
$$
\stream{M} \id_A\,(\mcons\,m)
= \mcons\,(M\,(\id_A\times \stream{M} \id_A)\,m)
$$
The recursive occurrence of the function $(\stream{M} \id_A)$ occurs as argument of the functor $M$, which in turn is guarded by the constructor $\mcons$.
This guarantees the fulfilment of the {\em guardedness by constructors} constraint, so we can apply the {\em coinduction hypothesis} allowing us to apply the Lemma's statement to this occurrence:
$$
\begin{array}{rl@{\qquad}l}
\cdots
& {}= \mcons\,(M\,(\id_A\times \id_{(\stream{M} A)})\,m)
  & \mbox{coinduction hypothesis}\\
& {}= \mcons\,(\id_{M\,(A\times \stream{M} A)}\,m)
  & \mbox{functoriality of }\times\mbox{ and }M\\
& {}= \mcons\,m
\end{array}
$$
This concludes the proof that the statement's identity holds for all objects in canonical constructor form.
We have recursively used the statement as coinductive hypothesis only on a subterm guarded by constructor.
This validates this proof as a correct application of coinduction.
\end{proof}

\begin{lemma}\label{lemma:functor_comp}
The composition functor law holds for monadic streams:

If $f:A\rightarrow B$ and $g:B\rightarrow C$, then
$$
\stream{M} (g\comp f) = (\stream{M} g) \comp (\stream{M} f)
$$
\end{lemma}
\begin{proof}
We apply the left-hand side function to an $M$-monster in constructor form and simplify according to the definition of $\stream{M}$ mapping:
$$
\stream{M} (g\comp f)\,(\mcons\,m)
  = \mcons\,(M\,((g\comp f)\times \stream{M} (g\comp f))\,m)
$$
The recursive occurrence of the function $\stream{M} (g\comp f)$ occurs as argument of the functor $M$, which in turn is guarded by the constructor $\mcons$.
We can therefore apply the coinduction hypothesis to it:
$$
\begin{array}{l@{\quad}l}
\cdots\\
{}= \mcons\,(M\,((g\comp f)\times ((\stream{M} g) \comp (\stream{M} f)))\,m)
  & \mbox{coinduction hypothesis}\\ 
{}= \mcons\,(M\,((g\times \stream{M} g) \comp (f\times \stream{M} f))\,m)
  & \mbox{functoriality of }\times\\ 
{}= \mcons\,((M\,(g\times \stream{M} g) \comp M\,(f\times \stream{M} f))\,m)
  & \mbox{functoriality of }M\\ 
{}= \mcons\,(M\,(g\times \stream{M} g)\, (M\,(f\times \stream{M} f)\,m))
  & \mbox{definition of composition}\\ 
{}= \stream{M} g\, (\mcons\,(M\,(f\times \stream{M} f)\,m))
  & \mbox{definition of }\stream{M}\mbox{ mapping}\\ 
{}= \stream{M} g\, (\stream{M} f\,(\mcons\,m))
  & \mbox{definition of }\stream{M}\mbox{ mapping}\\ 
{}= ((\stream{M} g) \comp (\stream{M} f))\,(\mcons\,m)
  & \mbox{definition of composition}
\end{array}
$$
\end{proof}

We can sum up these results by stating that the monster operator is a functor if the underlying operator is.

\begin{theorem}
If $M$ is a functor, $\stream{M}$ is also a functor.
\end{theorem}
