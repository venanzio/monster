\documentclass[a4paper,UKenglish,cleveref, autoref, thm-restate]{lipics-v2021}

\bibliographystyle{plainurl}

\usepackage{amssymb,amsmath,amsthm}
\usepackage{pstricks,pst-node}

\title{Monadic Streams are Applicative but not Monadic}
\author{Venanzio Capretta}{Nottingham University, UK}
       {venanzio.capretta@nottingham.ac.uk}{}{}
\author{Christopher Purdy}{Nottingham University, UK}
       {psycjpu@exmail.nottingham.ac.uk}{}{}

\authorrunning{Venanzio Capretta and Christopher Purdy}

\Copyright{Venanzio Capretta and Christopher Purdy}
\ccsdesc[500]{Theory of computation~Type theory}

\keywords{monadic streams, applicative functor, monad}
\usepackage{notations}

%\newtheorem{definition}{Definition}
%\newtheorem{theorem}[definition]{Theorem}
%\newtheorem{proposition}[definition]{Proposition}
%\newtheorem{lemma}[definition]{Lemma}
%\newtheorem{example}[definition]{Example}
%\newtheorem{conjecture}[definition]{Conjecture}

\begin{document}

\maketitle

\begin{abstract}
A monadic stream is a potentially infinite sequence of values in which every element triggers a monadic actions.
We prove that if the underlying functor is applicative, monadic streams are also applicative.
On the other hand, in general monadic streams don't form a monad, even if the underlying functor is a monad.
\end{abstract}

\section{Introduction}

A monadic stream is a sequence of values in which every element is obtained by triggering a monadic action.
If $\sigma$ is such a stream, it will consist of an action for a certain monad $M$ that, when executed, will return a head (first element) and a tail (continuation of the stream).
This process can be continued in a non-well-founded way: streams constitute a coinductive type.

Formally the type of streams over a monad $M$ (let's call them {\em $M$-monsters}) with elements of type $A$ is defined, with an Agda-like notation \cite{agda}, as:

$$
\begin{array}[t]{l}
\codata\;
\stream{M}\,A:\set\\
\quad \mcons_M: M\,(A\times \stream{M}\,A)\rightarrow\stream{M}\,A
\end{array}
$$

Categorically, we can see this type as the {\em final coalgebra} of the functor $F_M\,X = M\,(A\times X)$.
The final coalgebra does not necessarily exist for every $M$, 
but it does for most of the commonly used monads, specifically for those that are container functors \cite{AAG:2005}.

The definition of $M$-monsters is very close to that of {\em cofree (or iterative) comonad}, which can be seen as the type of $M$-monsters with a pure 
leading value \cite{AAMV:2003,CUV:2006}.

The monadic streams definition is a type operator $\stream{M}$ that maps a type $A$ to the type of $M$-monsters with elements of type $A$.
Instantiating $M$ with some of the most well-known monads leads to versions of known data types or to interesting new constructs.

If we instantiate $M$ with the identity monad, we obtain the type of pure streams.
Its usual definition is the following:

$$
\begin{array}[t]{l}
\codata\;
\strsym\,A:\set\\
\quad (\scons): A\rightarrow \strsym\,A \rightarrow\strsym\,A
\end{array}
$$

An element of $(\strsym\,A)$ is an infinite sequence of elements of $A$: $a_0 \scons a_1\scons a_2\scons \cdots$.

If we instantiate $M$ with the $\maybe$ monad we obtain the type $\stream{\maybe}\,A$, equivalent to the type of lazy lists $\lst{A}$.
The $\maybe$ monad is a functor that adds an extra element to the argument type:
$\maybe\,X$ contains copies of each element $x:X$, denoted by $\just\,x$, plus a {\em empty} element $\nothing$.
So $\maybe\,X \cong X+1$.
The single constructor $\mcons_\maybe: \maybe\,(A\times \stream{\maybe}\,A)\rightarrow\stream{\maybe}\,A$ is equivalent to two constructors (for $\nothing$ and $\just$):

$$
\begin{array}[t]{l}
\codata\;
\lst{A}:\set\\
\quad (\scons): A\times \lst{A}\rightarrow\lst{A}\\
\quad \nil: \lst{A}.
\end{array}
$$

This means that an element of $\lst{A}$ is either an empty sequence $\nil$ or a non-empty sequence $a\scons \sigma$ where $a:A$ and $\sigma$ is recursively an element of $\lst{A}$.
Since this is a coinductive type, the constructor $(\scons)$ can be applied an infinite number of times.
Therefore $\lst{A}$ is the type of finite and infinite sequences.

Another example is when the underlying monad is $M = \lstsym $ itself.
In this case each entry in the stream is a list of pairs of heads and tails.
This is equivalent to trees of arbitrary branching degrees (finite branches if we use only finite lists, but also countably infinite branches if we use lazy lists).
Since the type is coinductive, the trees can be non-well-founded, that is, they may be infinitely deep.

If we choose $M$ to be the {\em state transformer} monad, $M$-monsters are state machines that, at every step, produce an output value that depends on an underlying state and change the state itself.
The state transformer monad is defined as $\state_S\,A = S\rightarrow A\times S$, where $S$ is the type of underlying states.
We use the notation $\stmon{S}\,A = \stream{\state_S}\,A$ to denote state monsters.
They will be useful to construct the counterexample to the monad laws in Section \ref{sec:monad}.

Finally, in Haskell user interaction and other system effects are encapsulated in the $\io$ monad. An $\io$ action produces a value that possibly depends and triggers effects. $\io$-monsters are interactive processes that continuously interface with the external world.

A companion paper that we will publish later will present the Haskell monster library that we developed and describe several of its applications.

It is important to make two observations about the underlying ``monad'' $M$.

First, $M$ does not need to be a monad for the definition to make sense. 
In fact we will obtain several interesting results when $M$ satisfies weaker conditions, for example being an applicative functor.
So we will take $M$ to be any type operator (but see second observation) and we will explicitly state what properties we assume about it.
The most important instances are monads and it is convenient to use the facilities of monadic notation in programming and monad theory in reasoning.

The second observation is that it is not guaranteed in general that the $\codata$ type is well-defined.
Haskell will accept the definition when $M$ is any operator, but mathematically the type is well defined only when $F_M\,X = M\,(A\times X)$ is a functor with a final coalgebra.
As stated above, this is the case if $M$ is a container functor and we assume that it is for the rest of the article.

Functions with $M$-monsters as codomain can be defined by explicitly giving a coalgebra.
In practical programming it is more convenient to define function by {\em corecursive equations} that satisfy the property of {\em guardedness by constructors}:
a corecursive equations is accepted if the right-hand side is a term with a constructor on top and recursive calls occurring only as direct arguments of that constructor.

As an example, here is a function that generates a state machine from a natural number:

$$
\begin{array}{l}
\frominc : \nat \rightarrow \stmon{\nat}\,\nat \\
\frominc\, n = \mcons\, (\lambda s. \pair{\pair{s+n}{\frominc\, (n + 1)}}{s + 1})
\end{array}
$$

Given an input $n$, this defines a state-monster that reads the current state $s$, returns its as output the value $s+n$, increments the state by 1, and continues recursively by calling itself on input $n+1$.
In the end it will produce the infinite sequence $s+n \scons s+n+1 \scons s+n+2 \scons \cdots$.
The justification of the soundness of the definition is in the fact that the recursive call $\frominc\, (n + 1)$ occurs under the guard of the constructor $\mcons $ and is used to generate the recursive sub-stream of the main output value.
The fact that the input $n+1$ is larger than the original input $n$ is irrelevant to corecursive definitions (contrary to what happens in inductive definitions): what matter is that the first element of the stream and the new state are produced before the call is executed.

Another example is the function from monsters to monsters that increases every element by one.
This is a function polymorphic on $M$ that only assumes that $M$ is a functor.

$$
\begin{array}{l}
\incr : \stream{M}\,\nat \rightarrow \stream{M}\,\nat\\
\incr\,(\mcons\,m) = \mcons\,(M\,(\lambda \pair{k}{\sigma}. \pair{k+1}{\incr\,\sigma})\,m)
\end{array}
$$

Here the pattern-matching variable $m$ has type $M\,(\nat\times \stream{M}\,\nat)$.
We map onto it the function that increases the output value by one and recursively applies $\incr$ to the tail.
This recursive occurrence is justified because it is guarded by $\mcons$ and mapped to the direct subterms inside the monad action.
By the way,
after we prove that $\stream{M}$ is a functor whenever $M$ is in Section \ref{sec:functor}, and that it is applicative whenever $M$ is in Section \ref{sec:applicative},
the definition can be simplified and made more intuitive in the following ways, respectively:

$$
\incr\,\sigma = \stream{M}\,(+1)\,\sigma
\qquad
\incr\,\sigma = \apure\,(+1) \appl \sigma
$$

In the proofs of our results we will make frequent use of the technique of proof by {\em coinduction} on monsters.
Inductive types comply with an {\em induction principle}, which states that we can prove statements about them by bottom-up recursion on their structure.
Dually, coinductive types comply with a {\em coinduction principle}, which states that we can prove equalities of their elements by top-down {\em co-recursion} on their structure.

Let us illustrate the idea with the example of pure streams:
Suppose that we want to prove that two streams $\sigma_0$ and $\sigma_1$ are equal.
Since streams are infinite sequences of elements, that will require proving equality of their entries in corresponding positions: if $\sigma_0 = a_0\scons a_1\scons a_2\scons \cdots$ and  $\sigma_1 = b_0\scons b_1\scons b_2\scons \cdots$, we must prove $a_0 = b_0$, $a_1 = b_1$, $a_2=b_2$, and so on (we assume equality on streams is extensional).
This requires an infinite sequence of equalities to prove, and clearly we cannot produce all of these explicitly.
However, the proof of equality can be seen itself as a stream of proofs of equalities of each pair of elements in the same positions.
We can use the same principle of guarded recursion to generate all the proofs: we recursively assume that we can prove the equality of the tail streams and we only need to give explicitly the equality of the heads.

In practice the coinduction principle is applied by a corecursive proof that can invoke the statement to be proved under certain structural restrictions.
When proving the equality of two given terms, we can appeal to the statement that we want to prove, as a {\em coinduction hypothesis}, as long as it is {\em guarded by constructors} in the sense that it is only deployed to prove the equality of direct components of the given terms.

As an example,
let us prove a characterization of the function $\frominc$ defined above.
\begin{lemma}
$$
\forall n:\nat. \frominc\,(n+1) = \incr\,(\frominc\,n)
$$
\end{lemma}
\begin{proof}
The proof consists of a sequence of computational steps.
Most of them consists in folding and unfolding definitions and using basic properties of the operators involved, specifically functoriality of the state operator.
The one unusual step is the {\em coinduction hypothesis} in which we allow ourselves to use an instance of the very statement we are proving.

$$
\begin{array}{ll}
\frominc\,(n+1)\\
{}= \mcons\, (\lambda s. \pair{\pair{s+n+1}{\frominc\, (n + 2)}}{s + 1})
  & \mbox{definition of }\frominc\\
{}= \mcons\, (\lambda s. \pair{\pair{s+n+1}{\incr\,(\frominc\,(n+1))}}{s + 1})
  & \mbox{\bf coinduction hypothesis}\\
{}= \mcons\,(
    \begin{array}[t]{l}
    \state_\nat\,(\lambda \pair{k}{\sigma}. \pair{k+1}{\incr\,\sigma})\\
    \quad (\lambda s. \pair{\pair{s+n}{\frominc\,(n+1)}}{s + 1}))
    \end{array}
  & \mbox{functoriality of }\state_\nat\\
{}= \incr\,(\mcons\,(\lambda s. \pair{\pair{s+n}{\frominc\,(n+1)}}{s + 1}))
  & \mbox{definition of }\incr\\
{}= \incr\,(\frominc\,n)
  & \mbox{definition of }\frominc
\end{array}
$$

The coinduction step invokes the main statement of the lemma to rewrite a subterm.
This is apparently circular: we assume the truth of the statement while we are proving it.
However, the occurrence of the hypothesis is guarded by the constructor $\mcons$ and is applied to the recursive argument inside the monadic action.
The monadic action itself is unchanged a part from this recursive application of the hypothesis.
This guarantees the the top levels of the two terms are equal.
It is therefore sound to invoke the hypothesis to prove the lower levels.
\end{proof}

We will use this style of reasoning repeatedly in the article.
Most of the proof consist in equational rewriting, with most steps consisting in term reduction, application of known laws and of previously proved lemmas.
There will usually be a single step justified by {\em coinduction hypothesis}: it will always occur under the guard of a constructor and be mapped inside the monadic action.

See previous survey work \cite{capretta:2011} for an overview of the theory of final coalgebras, coinductive types, corecursive definitions, and proof by coinduction.

The main goal of this article is to study the algebraic properties of monsters, in the form of membership to operator classes.
Is $\stream{M}$ in general a functor, an applicative functor, a monad?
What assumptions do we need to make on $M$ to obtain membership of these classes?
In summary our results are:
\begin{itemize}
\item If $M$ is a functor, $\stream{M}$ is also a functor;
\item If $M$ is an applicative functor, $\stream{M}$ is also an applicative functor;
\item $\stream{M}$ is not a monad is general, even when $M$ is a monad.
\end{itemize}

The proof of the first result is straightforward and unproblematic.
We give the proofs of the functor laws in Section \ref{sec:functor} anyway, as a warm-up in the use of coinduction arguments that will be illuminating in view of the more complex proofs to come.
The proofs of the applicative laws, on the other hand, are surprisingly challenging and require the definition of auxiliary notation and several ancillary lemmas.
Finally, a counterexample using the state transition monad shows that monsters do not form a monad in general. Our counterexample assumes that the join operation takes the {\em diagonal} of a stream of streams.
This is one possible attempt at a definition, but since the join operation on pure streams ($\Id$-monsters) operates by taking the diagonal, the principle of polymorphic parametricity suggests that no alternative is possible.

These results are important theoretically, they shed light on the abstract properties of monsters, and practically, they allow programmers to use specific notations and extensive libraries for these classes.

\section{Monsters are Functors}\label{sec:functor}

In this section we assume that $M$ is a functor.
To show that $\stream{M}$ is also a functor, we have to define its behaviour on morphisms: if $f:A\rightarrow B$, then we must define how $f$ maps on monadic streams:
$$
\begin{array}{l}
\stream{M}\,f : \stream{M,A} \rightarrow \stream{M,B}\\
\stream{M}\,f\,(\mcons\,m) = \mcons\,(M\,(f\times \stream{M}\,f)\,m)
\end{array}
$$

This definition complies with the {\em guarded-by-constructors} discipline: the recursive call to $(\stream{M}\,f)$ is mapped to the recursive substreams by the functorial application of the functor $M\,(A \times -)$.
That is: $(\stream{M}\,f)$ will be recursively applied only at the recursive positions inside the shape of $m$.
There are also applications of $f$ to the first element (of type $A$) of the pairs in the $M$-position: this is non-recursive, and therefore not problematic.


We can now prove that the functor laws are satisfied by $\stream{M}$: its functorial mapping preserves identity and composition.
The proofs are straightforward applications of definitions and the functoriality of $M$ and $\times$, except for the use of coinduction;
we are allowed to invoke the laws themselves in their proofs, as long as we use them only in the direct recursive subterms of the $\mcons$ constructor, that is, in the {\em positions} for the container $M\,(A\times -)$.

\begin{lemma}\label{lemma:functor_id}
The identity functor law holds for monadic streams:
$$
\stream{M}\,\id_A = \id_{\stream{M,A}}
$$
\end{lemma}
\begin{proof}
We apply the left-hand side function to an $M$-monster in constructor form:
$$
\begin{array}{ll}
\stream{M}\,\id_A\,(\mcons\,m)\\
{}= \mcons\,(M\,(\id_A\times \stream{M}\,\id_A)\,m)
  & \mbox{by definition}\\
{}= \mcons\,(M\,(\id_A\times \id_{\stream{M,A}})\,m)
  & \mbox{by coinduction hypothesis}\\
{}= \mcons\,(M\,(\id_{A\times \stream{M,A}})\,m)
  & \mbox{by functoriality of }\times\\
{}= \mcons\,(\id_{M\,(A\times \stream{M,A})}\,m)
  & \mbox{by functoriality of }M\\
{}= \mcons\,m
\end{array}
$$
\end{proof}

\begin{lemma}\label{lemma:functor_comp}
The composition functor law holds for monadic streams:

If $f:A\rightarrow B$ and $g:B\rightarrow C$, then
$$
\stream{M}\,(g\comp f) = (\stream{M}\,g) \comp (\stream{M}\,f)
$$
\end{lemma}
\begin{proof}
Let's again apply the left-hand side function to an $M$-monster in constructor form:
$$
\begin{array}{ll}
\stream{M}\,(g\comp f)\,(\mcons\,m)\\
{}= \mcons\,(M\,((g\comp f)\times \stream{M}\,(g\comp f))\,m)
  & \mbox{by definition of }\stream{M}\mbox{ mapping}\\
{}= \mcons\,(M\,((g\comp f)\times ((\stream{M}\,g) \comp (\stream{M}\,f)))\,m)
  & \mbox{by coinduction hypothesis}\\ 
{}= \mcons\,(M\,((g\times \stream{M}\,g) \comp (f\times \stream{M}\,f))\,m)
  & \mbox{by functoriality of }\times\\ 
{}= \mcons\,((M\,(g\times \stream{M}\,g) \comp M\,(f\times \stream{M}\,f))\,m)
  & \mbox{by functoriality of }M\\ 
{}= \mcons\,(M\,(g\times \stream{M}\,g)\, (M\,(f\times \stream{M}\,f)\,m))
  & \mbox{by definition of composition}\\ 
{}= \stream{M}\,g\, (\mcons\,(M\,(f\times \stream{M}\,f)\,m))
  & \mbox{by definition of }\stream{M}\mbox{ mapping}\\ 
{}= \stream{M}\,g\, (\stream{M}\,f\,(\mcons\,m))
  & \mbox{by definition of }\stream{M}\mbox{ mapping}\\ 
{}= ((\stream{M}\,g) \comp (\stream{M}\,f))\,(\mcons\,m)
  & \mbox{by definition of composition}
\end{array}
$$
\end{proof}

We can sum up these results by stating that the monster operator is a functor if the underlying `monad' is (remember that we are not actually assuming that $M$ is a monad yet, but just a type operator).

\begin{theorem}
If $M$ is a functor, $\stream{M}$ is also a functor.
\end{theorem}

\section{Monsters are Applicatives}\label{sec:applicative}

Applicative functors \cite{mcbride/paterson:2008}
extend the mapping operation by allowing function sequencing under the functor.
The Applicative class has two methods: $\apure$, that injects single values into the functor, and $\appl$, that applies functions under the functor:
$$
\begin{array}{l}
\apure : A \rightarrow M\,A\\
(\appl): M\,(A\rightarrow B) \rightarrow M\,A \rightarrow M\,B
\end{array}
$$
They must satisfy the four applicative laws given below (after we introduce some notation).

A typical use of applicative functors is to apply a function of many arguments to several applicative values.
If $g:A_0\rightarrow A_1 \rightarrow \cdots \rightarrow A_n \rightarrow B$ and $m_0:M\,A_0, m_1:M\,A_1, \ldots, m_n:M\,A_n$ , then:
$$
\apure\,g \appl m_0 \appl m_1 \appl \cdots \appl m_n : M\,B
$$

In particular, if $g$ is an infix binary operator $(\oplus) : A \rightarrow B \rightarrow C$, we can lift it to $M$-objects ($\pairing$ is the pairing operator, $\pairing\,a\,b = \pair{a}{b}$, and $\pairop{\oplus}$ operates on pairs, $\pairop{\oplus}\,\pair{a}{b} = a \oplus b$):
$$
m_a \alift{\oplus} m_b 
= \apure\,(\oplus) \appl m_a \appl m_b
= \apure\,(\pairop{\oplus}) \appl (\apure\,\pairing\appl m_a\appl m_b)
$$
\vcomm{I haven't yet proved the right-hand side}

Here are the applicative functor laws for $M$, where
 $a:A$, $f:A\rightarrow B$,
$m_a:M\,A$, $m_f:M\,(A\rightarrow B)$, $m_g:M\,(B\rightarrow C)$.
$$
\begin{array}{l@{\qquad}l}
\apure\,\id \appl m_a = m_a
  & \mbox{Identity} \\
\apure\,f \appl \apure\,a = \apure\,(f\,a)
  & \mbox{Homomorphism} \\
m_f \appl \apure\,a = \apure\,(\lambda f. f\,a) \appl m_f
  & \mbox{Interchange} \\
m_g \appl (m_f \appl m_a) = (m_g \alift{\comp} m_f) \appl m_a
  & \mbox{Composition}
\end{array}
$$

A consequence of the Homomorphism and Composition Laws is that sequential application of functions can be reduced to a single application of the composition;
if $g:B\rightarrow D$ and $h:A_1\rightarrow A_2 \rightarrow C$, then:
$$
\begin{array}{l}
\apure\,g \appl (\apure\, f \appl m_a) = \apure\, (g\comp f) \appl m_a\\
\apure\,g \appl (\apure\, h \appl m_{a_1} \appl m_{a_2})
  = \apure\, (\lambda a.\lambda b. g\,(h\,a\,b)) \appl m_{a_1} \appl m_{a_2}
\end{array}
$$


Functoriality of $M$ follows from these laws, with functorial mapping defined as
$
M\,f\,m_a = \apure\,f \appl m_a.
$
The homomorphism law implies that
$
M\,f\,(\apure\,a) = \apure\,(f\,a).
$
We can further characterize the applicative operator lifting as:
$$
m_a \alift{\oplus} m_b 
= M\,(\pairop{\oplus})\,(\apure\,\pairing\appl m_a\appl m_b)
$$


We will show that $\stream{M}$ is also applicative.
In order to define the methods, we need some auxiliary functions.
First of all, simplified versions of $\mcons$: appending either an $M$-action or a pure value in front of a monster.
The first works when $M$ is just a functor, the second when it is an applicative:
$$
\begin{array}{l}
(\fcons): M\,A \rightarrow \stream{M}\,A \rightarrow \stream{M}\,A\\
m \fcons \sigma = \mcons\, (M\,(\lambda a. \langle a, \sigma\rangle)\,m)
\end{array}\qquad
\begin{array}{l}
(\acons): A \rightarrow \stream{M}\,A \rightarrow \stream{M}\,A\\
a \acons \sigma = (\apure\, a) \fcons \sigma
\end{array}
$$

The $\apure$ method for monsters repeats the same element forever:
$$
\begin{array}{l}
\apure : A \rightarrow \stream{M}\,A\\
\apure\,a = a \acons \apure\,a
\end{array}
$$
To check that the recursive occurrence of $\apure\,a$ satisfies the guardedness condition we just unfold the definitions (we write the functor as a subscript to clarify which version of $\apure$ is used):
$$
\begin{array}{ll}
\apure_{\stream{M}}\,a 
& {} = a \acons \apure_{\stream{M}}\,a 
     = (\apure_M\, a) \fcons \apure_{\stream{M}}\,a \\
& {} = \mcons\, (M\,(\lambda a. \langle a, \apure_{\stream{M}}\,a \rangle)\, (\apure_M\, a))
\end{array}
$$
The recursive occurrence of $\apure_{\stream{M}}\,a$ is guarded by $\mcons$ and the functorial application of $M$.
The occurrence of $(\apure_M\, a)$ on the other hand. is not recursive and therefore unproblematic.

The homomorphism law for $M$ provides a further simplification.
\begin{lemma}\label{lemma:monster_pure} 
$$
\apure_{\stream{M}}\,a = \mcons\,(\apure_M\,\langle a, \apure_{\stream{M}}\,a\rangle)
$$
\end{lemma}

The previous discussion shows that the second arguments of $(\fcons)$ and $(\acons)$, when definitions are expanded, are guarded by the constructor $\mcons$.
Therefore, in future definitions and proofs, we allow ourselves to give corecursive definitions as that for $\apure\,a$ without justifying them by expansion.

In the following statements and proofs, we often need to write terms that combine the use of $\apure$ on $M$ and on $\stream{M}$.
This could be confusing, so we explicitly write the functors as subscript: $\apure_M$ and $\apure_{\stream{M}}$.
We also explicitly write the type of identity functions as a subscript.
We consistently use $\sigma$ with subscripts to denote monsters and $m$ with subscripts to denote $M$-actions.
We adopt the convention that $\apure$ binds stronger than any of the infix operators that we use, specifically $\appl$, $\alift{\comp}$ and $\alift{\pappl}$ (defined below).


We define the function application method by mapping straight function application on the heads and recursive calls on the tails through functorial and applicative lifting:

$$
\begin{array}{l}
(\appl) : \stream{M}\,(A\rightarrow B) \rightarrow \stream{M}\,A \rightarrow \stream{M}\,B\\
(\mcons\,m_f) \appl (\mcons\,m_a)
= \mcons\,(m_f \alift{\pappl} m_a)\\
\qquad \where\;
\langle f,\phi\rangle \pappl \langle a,\sigma\rangle 
= \langle f\,a, \phi \appl \sigma \rangle
\end{array}
$$

This definition recursively applies $\appl$ indirectly in the second components of the arguments of the $\pappl$ operator.
This is lifted to $\alift{\pappl}$, which distributes down through the components of the applicative values $m_f$ and $m_a$, and finally guarded by the constructor $\mcons$.
This guarantees the soundness of the definition according to the {\em guardedness-by-constructors} criterion.

We aim to show that these definitions satisfy the applicative laws.
We assume that the laws hold for the base functor $M$ and we wish to prove them for the monster functor $\stream{M}$.

The new operator $\alift{\pappl}$ has a direct simplification on pure values, which will be very useful in the proofs of the laws.
\begin{lemma}\label{lemma:pappl}
If $f:A\rightarrow B$, $m_f:\stream{M}\,(A\rightarrow B)$, $a:A$, $m_a:\stream{M}\,A$, then:

$$
\apure_M\,\langle f, m_f\rangle \alift{\pappl} \apure_M\,\langle a, m_a\rangle = \apure_M \langle f\,a, m_f \appl m_a \rangle
$$

\end{lemma}
\begin{proof}

$$
\begin{array}{ll}
\apure_M\,\langle f, m_f\rangle \alift{\pappl} \apure_M\,\langle a, m_a\rangle \\
{}= \apure_M\,(\pappl) \appl \apure_M\,\langle f, m_f\rangle \appl \apure_M\,\langle a, m_a\rangle
  & \mbox{definition of }\alift{-}\\
{}= \apure_M\,(\langle f, m_f\rangle \pappl \langle a, m_a\rangle)
  & \mbox{the homomorphism law for }M\\
{}= \apure_M\,\langle f\,a, m_f \appl m_a \rangle
  & \mbox{definition of }\pappl
\end{array}
$$
\end{proof}

\begin{lemma}
Monsters satisfy the applicative identity law:

$$
\apure_{\stream{M}}\,\id_A \appl \sigma_a = \sigma_a
$$

\end{lemma}
\begin{proof}
Assume that $\sigma_a$ has the canonical form:
$\sigma_a = \mcons\,m$ where $m:M\,(A\times \stream{M}A).$


Most of the proof is by straightforward equational reasoning, unfolding definitions and using the applicative laws for $M$.

The only unusual step is the application of the {\em coinduction hypothesis}.
Recall that this means that we can assume that the statement to prove is already true, as long as it is invoked only on direct arguments of the monster constructor.

$$
\begin{array}{ll}
\apure_{\stream{M}}\,\id_A \appl \sigma_a \\
{}= (\id_A \acons (\apure_{\stream{M}}\,\id_A)) \appl \sigma_a
  & \mbox{definition of }\apure \\
{}= ((\apure_M\,\id_A) \fcons (\apure_{\stream{M}}\,\id_A)) \appl \sigma_a
  & \mbox{definition of }\acons \\
{}= (\mcons\,(M\,(\lambda f. \langle f, \apure_{\stream{M}}\,\id_A\rangle)\,(\apure_M\,\id_A))) \appl \sigma_a
  & \mbox{definition of }\fcons \\
{}= (\mcons\,(\apure_M\,\langle \id_A, \apure_{\stream{M}}\,\id_A \rangle)) \appl \sigma_a
  & \mbox{interchange law for }M \\
{}= (\mcons\,(\apure_M\,\langle \id_A, \apure_{\stream{M}}\,\id_A\rangle)) \appl (\mcons\,m)
  & \mbox{canonical form of }\sigma_a\\
{}= \mcons\,(\apure_M\,\langle \id_A, \apure_{\stream{M}}\,\id_A\rangle)) \alift{\pappl}  m)
  & \mbox{definition of }\appl \\
{}= \mcons\,(\apure_{M}\,(\pappl) \appl \apure_M\,\langle \id_A, \apure_{\stream{M}}\,\id_A\rangle \appl  m)
  & \mbox{definition of }\alift{-} \\
{}= \mcons\,(\apure_M\,((\pappl)\,\langle \id_A, \apure_{\stream{M}}\,\id_A\rangle) \appl  m)
  & \mbox{homomorphism law for }M\\
{}= \mcons\,(\apure_M\,(\lambda \langle a,\sigma_a\rangle. \langle \id_A, \apure_{\stream{M}}\,\id_A\rangle \pappl \langle a,\sigma_a\rangle) \appl m)
  & \mbox{expansion of the section}\\
{}= \mcons\,(\apure_M\,(\lambda \langle a,\sigma_a\rangle. \langle \id_A\,a, \apure_{\stream{M}}\,\id_A \appl \sigma_a\rangle) \appl m)
  & \mbox{definition of }\pappl\\
{}= \mcons\,(\apure_M\,(\lambda \langle a,\sigma_a\rangle. \langle a, \sigma_a\rangle) \appl m)
  & \mbox{coinduction hypothesis}\\
{}= \mcons\,(\apure_M\,\id_{A\times\stream{M}A} \appl m)
  & \mbox{trivial}\\
{}= \mcons\,m = \sigma_a
  & \mbox{identity law for }M
\end{array}
$$

In the step that makes use of the coinduction hypothesis, we circularly use the lemma's statement that $\apure_{\stream{M}}\,\id_A \appl \sigma_a = \sigma_a$.
The right-hand side occurs under an application of the constructor $\mcons$, inside an application of a pure value. 
This structural positioning of the term ensure that the application of the coinductive hypothesis is sound.
\end{proof}

\begin{lemma}
Monsters satisfy the applicative homomorphism law:

$$
\apure_{\stream{M}}\,f \appl \apure_{\stream{M}}\,a = \apure\,(f\,a)
$$

\end{lemma}
\begin{proof}

$$
\begin{array}{ll}
\apure_{\stream{M}}\,f \appl \apure_{\stream{M}}\,a\\
{}= (\mcons\,(\apure_M\,\langle f, \apure_{\stream{M}}\,f\rangle)) \appl
    (\mcons\,(\apure_M\,\langle a, \apure_{\stream{M}}\,a\rangle))
  & \mbox{Lemma \ref{lemma:monster_pure}}\\
{}= \mcons\,(\apure_M\,\langle f, \apure_{\stream{M}}\,f\rangle \alift{\pappl}
             \apure_M\,\langle a, \apure_{\stream{M}}\,a\rangle)
  & \mbox{definition of }\appl\\
{}= \mcons\,(\apure_M\,\langle f\, a, \apure_{\stream{M}}\,f \appl \apure_{\stream{M}}\,a \rangle))
  & \mbox{Lemma }\ref{lemma:pappl}\\
{}= \mcons\,(\apure_M\,\langle f\,a, \apure_{\stream{M}}\,(f\,a)\rangle)
  & \mbox{coinduction hypothesis}\\
{}= \apure_{\stream{M}}\,(f\,a)
  & \mbox{lemma }\ref{lemma:monster_pure}
\end{array}
$$
\end{proof}


% Proof of interchange law with M applicative
The proofs of the following technical lemmas are included in the Appendix.
\vcomm{maybe just say that they are easy equational proofs, no Appendix}

\begin{lemma}\label{lemma:pure_lift}
If $a:A$, $b:B$ and $(\oplus) : A \rightarrow B \rightarrow C$, then:
$$
\apure_M\, a \alift{\oplus} \apure_M\, b = \apure_M\, (a \oplus b)
$$
\end{lemma} 

\begin{lemma}\label{lemma:applicative_flip}
If $f : A \rightarrow B \rightarrow C$, $m_a : M\, A$ and $b : B$, then:
$$
\apure_M\, f \appl m_a \appl \apure_M\, b = \apure_M\, (\lambda a . f\, a\, b) \appl m_a
$$
\end{lemma}

This lemma is extremely useful in the following proofs of the last two applicative laws.

\begin{lemma}
Monsters satisfy the applicative interchange law.
 If $\sigma_f: \stream{M}(A\rightarrow B)$ and $a:A$, then:

$$
\sigma_f \appl \apure_{\stream{M}}\,a = \apure_{\stream{M}}\,(\lambda f. f\,a) \appl \sigma_f
$$

\end{lemma}
\begin{proof}
Assume that $\sigma_f = \mcons\,m_f$ for some $m_f:M\,((A\rightarrow B)\times \stream{M}\,(A\rightarrow B))$.

$$
\begin{array}{ll}
\sigma_f \appl \apure_{\stream{M}}\,a = (\mcons\,m_f) \appl \apure_{\stream{M}}\,a \\
{}= (\mcons\,m_f) \appl (\mcons\,(\apure_M\,\langle a, \apure_{\stream{M}}\,a\rangle))
  & \mbox{Lemma \ref{lemma:monster_pure}}\\
{}= \mcons\,(m_f \alift{\pappl} \apure_M\,\langle a, \apure_{\stream{M}}\,a\rangle)
  & \mbox{definition of }\appl\\
{}= \mcons\,(\apure_M\,(\pappl) \appl m_f \appl \apure_M\,\langle a, \apure_{\stream{M}}\,a\rangle)
  & \mbox{definition of }\alift{-}\\
{}= \mcons \, (\apure_M\,(\lambda \pair{f}{\phi} . \pair{f}{\phi} \pappl \pair{a}{\apure_{\stream{M}}\,a}) \appl m_f)
  & \mbox{Lemma \ref{lemma:applicative_flip}}\\
{}= \mcons\,(\apure_M\,(\lambda \pair{f}{\phi} . \pair{f a}{\phi \appl \apure_{\stream{M}}\,a}) \appl m_f)
  & \mbox{definition of } \pappl \\
{}= \mcons\,(\apure_M\,(\lambda \pair{f}{\phi} . \pair{f a}{\apure_{\stream{M}}\,(\lambda f . f\, a) \appl \phi}) \appl m_f)
  & \mbox{coinduction hypothesis} \\
{}= \mcons\,(\apure_M\,(\lambda \pair{f}{\phi}.
    \pair{\lambda f . f\, a}{\apure_{\stream{M}}\,(\lambda f . f\, a))}
    \pappl \pair{f}{\phi}
    ) \appl m_f)
  & \mbox{definition of } \pappl \\
{}= \mcons\,(\apure_M\,(\pappl) \appl \apure_M\,\langle \lambda f. f\,a, \apure_{\stream{M}}\,(\lambda f. f\,a)  \rangle \appl m_f)
  & \mbox{homomorphism law for }M \\
{}= \mcons\,(\apure_M\,\langle \lambda f. f\,a, \apure_{\stream{M}}\,(\lambda f. f\,a)  \rangle \alift{\pappl} m_f)
  & \mbox{definition of }\alift{-}\\
{}= (\mcons\,(\apure_M\,\langle \lambda f. f\,a, \apure_{\stream{M}}\,(\lambda f. f\,a)  \rangle)) \appl (\mcons\,m_f)
  & \mbox{definition of }\appl \\
{}= \apure_{\stream{M}}\,(\lambda f. f\,a) \appl (\mcons\,m_f)
  & \mbox{Lemma \ref{lemma:monster_pure}} \\
{} = \apure_{\stream{M}}\,(\lambda f. f\,a) \appl \sigma_f
\end{array}
$$
\end{proof}


To prove the final applicative law, we first need a few lemmas, and some new notation.
This law is much trickier than the previous ones, and requires many rewrite steps.

In the following we use repeatedly the types $A\times \stream{M}\,A$ and $M\,(A\times \stream{M}\,A)$, especially with $A$ being instantiated to different function types.
It is useful to use special notation for them:
$\pstream{A} = A\times \stream{M}\,A$, $\mstream{A} = M\,\pstream{A}$.
We use the notation $\pstr{a}$ for a generic element of $\pstream{A}$
and we allow ourselves to denote the first and second component of $\pstr{a}$ by $a$ and $\sigma_a$.
Elements of $\mstream{A}$ will be denoted by $m_a$.
We introduce notation for a {\em pure} value in the type $\pstream{A}$:
if $a:A$, we write $\ppair{a} = \langle a, \apure_{\stream{M}}\,a \rangle$.



We start with a little general lemma about {\em composition with compositions}.
The generic composition operator has type: $(\comp): (B\rightarrow C) \rightarrow (A\rightarrow B) \rightarrow A \rightarrow C$.
Suppose we want to compose it with a binary operator $(\oplus) : D\rightarrow (B\rightarrow C)$.
What would this higher composition, $(\comp) \comp (\oplus) : D \rightarrow (A\rightarrow B) \rightarrow A \rightarrow C$ be?
We will use the notation $\clift{\oplus}{A}$ for it.
\begin{lemma}\label{lemma:comp_comp}

$$
(\clift{\oplus}{A}) = \lambda d.\lambda f. \lambda a. d\oplus (f\,a)
$$

\end{lemma}
\begin{proof}
Just expand the definition of the two composition operators.
\end{proof}

We will in fact need to apply this {\em composition with compositions} operator twice: after producing $(\clift{\oplus}{A})$, we want in turn to compose it with the composition operator $(\comp): ((A\rightarrow B) \rightarrow (A\rightarrow C)) \rightarrow (X\rightarrow A\rightarrow B) \rightarrow X \rightarrow A \rightarrow C$ to produce the new operator $(\clift{\clift{\oplus}{A}}{X}): D\rightarrow (X\rightarrow A \rightarrow B) \rightarrow X \rightarrow A \rightarrow C$.
\begin{lemma}\label{lemma:comp_comp_comp}

$$
(\clift{\clift{\oplus}{A}}{X}) =
\lambda d. \lambda h. \lambda x.\lambda a. d\oplus (h\,x\,a)
$$

\end{lemma}
\begin{proof}
Apply Lemma \ref{lemma:comp_comp} twice and simplify.
\end{proof}

\begin{lemma}\label{lemma:pappl_comp_appl}
Let $m_g:\mstream{B\rightarrow C}$, $m_f:\mstream{A\rightarrow B}$, $m_a:\mstream{A}$, then:

$$
m_g \alift{\pappl} (m_f \alift{\pappl} m_a)
 = \apure\, (\lambda \pstr{g}. \lambda \pstr{f}. \lambda \pstr{a}. \pstr{g} \pappl (\pstr{f}\pappl \pstr{a})) \appl m_g \appl m_f \appl m_a
$$

\end{lemma}


\begin{lemma}\label{lemma:ppair}

$$
\apure_M\,\ppair{\comp}\, \alift{\pappl}\, m_g\, \alift{\pappl}\, m_f\, \alift{\pappl}\, m_a
= \apure_M\,(\lambda \pstr{g}. \lambda \pstr{f}. \lambda \pstr{a}. \ppair{\comp}\, \pappl\, \pstr{g} \, \pappl\, \pstr{f} \, \pappl\, \pstr{a})\, \appl\, m_g\, \appl\, m_f\, \appl\, m_a 
$$

\end{lemma}

\begin{lemma}
Monsters satisfies the applicative composition law:

$$
\sigma_g \appl (\sigma_f \appl \sigma_a)
 = 
(\sigma_g \alift{\comp} \sigma_f) \appl \sigma_a
$$
\end{lemma}
\begin{proof}
We prove this by rewriting both the left and right-hand-side of the law
until we reduce them to a common high-level structure.
They will then differ only by a subterm inside the leftmost application of $\apure$ in this structure.
This difference is resolved by proving that the two argument of the $\apure$ terms are equivalent, making use of the coinductive hypothesis.

As usual we assume that all monsters are in canonical form: $\sigma_g = \mcons\,m_g$, $\sigma_f = \mcons\,m_f$, $\sigma_a = \mcons\,m_a$.

First we rewrite the left-hand-side of the composition law. 

$$
\begin{array}{l}
\sigma_g \appl (\sigma_f \appl \sigma_a)
 = (\mcons\,m_g) \appl ((\mcons\,m_f) \appl (\mcons\,m_a))\\
{}= \mcons\,(m_g \alift{\pappl} (m_f \alift{\pappl} m_a))\\
  \qquad  \mbox{definition of}\appl\mbox{for monsters} \\
{}= \mcons\,(
\apure_M\, (\lambda \pstr{g}. \lambda \pstr{f}. \lambda \pstr{a}. \pstr{g} \pappl (\pstr{f}\pappl \pstr{a})) \appl m_g \appl m_f \appl m_a
)\\
  \qquad \mbox{Lemma \ref{lemma:pappl_comp_appl}}
\end{array}
$$

Next we rewrite the right-hand-side.

$$
\begin{array}{l}
(\sigma_g \alift{\comp} \sigma_f) \appl \sigma_a\\
{}= \apure_{\stream{M}}\,(\comp) \appl \sigma_g \appl \sigma_f \appl \sigma_a \\
  \qquad \mbox{definition of }\alift{-}\\
{}= \mcons\, (\apure_M\,\pair{(\comp)}{\apure_{\stream{M}}\,(\comp)} \alift{\pappl} m_g \alift{\pappl} m_f \alift{\pappl} m_a)\\
  \qquad \mbox{definition of}\appl\mbox{for monsters}\\
{}= \mcons\, (\apure_M\,\ppair{\comp} \alift{\pappl} m_g \alift{\pappl} m_f \alift{\pappl} m_a)\\
  \qquad \mbox{definition of }\ppair{-}\\
{}= \mcons\,(
\apure_M\,(\lambda \pstr{g}. \lambda \pstr{f}. \lambda \pstr{a}. \ppair{\comp} \pappl \pstr{g}  \pappl \pstr{f}  \pappl \pstr{a}) \appl m_g \appl m_f \appl m_a)\\
  \qquad \mbox{Lemma \ref{lemma:ppair}}
\end{array}
$$

To complete this proof, we need to show the following:

$$
\lambda \pstr{g}. \lambda \pstr{f}. \lambda \pstr{a}. \pstr{g} \pappl (\pstr{f}\pappl \pstr{a})
  = 
\lambda \pstr{g}. \lambda \pstr{f}. \lambda \pstr{a}. \ppair{\comp} \pappl \pstr{g}  \pappl \pstr{f}  \pappl \pstr{a}
$$

Let's assume the abstracted variables are in canonical form: $\pstr{g} = \pair{g}{\tau_g}$, $\pstr{f} = \pair{f}{\tau_f}$, $\pstr{a} = \pair{a}{\tau_a}$.

We use the coinduction hypothesis to prove the equality.
This is justified because the equality itself is embedded in an application of the constructor $\mcons$: in this context, $\tau_g$, $\tau_f$ and $\tau_a$ are guaranteed to be sub-streams of $\sigma_g$, $\sigma_f$ and $\sigma_a$ respectively, so the use of the coinductive hypothesis is sound.

$$
\begin{array}{ll}
\pstr{g}\, \pappl (\pstr{f} \pappl \pstr{a})
 = \pair{g}{\tau_g}\, \pappl (\pair{f}{\tau_f} \pappl \pair{a}{\tau_a}) \\
{}= \pair{g}{\tau_g}\, \pappl \pair{f\, a}{\tau_f \appl \tau_a}
& \mbox{definition of }\pappl \\
{}= \pair{g\, (f\, a)}{\tau_g \appl (\tau_f \appl \tau_a)}
& \mbox{definition of }\pappl \\
{}= \pair{(g \comp f)\, a)}{\tau_g \appl (\tau_f \appl \tau_a)}
& \mbox{definition of }\comp \\
{}= \pair{((\comp)\, g\, f)\, a}{\tau_g \appl (\tau_f \appl \tau_a)}
& \mbox{trivial} \\
{}= \pair{(((\comp)\, g)\, f)\, a}{(\tau_g \alift{\comp} \tau_f) \appl \tau_a}
& \mbox{coinduction hypothesis} \\
{}= \pair{(((\comp)\, g)\, f)\, a}{\apure_{\stream{M}}\, (\comp) \appl \tau_g \appl \tau_f \appl \tau_a}
& \mbox{definition of }\alift{-} \\
{}= \pair{((\comp)\, g)\, f}{\apure_{\stream{M}}\, (\comp) \appl \tau_g \appl \tau_f} \pappl  \pair{a}{\tau_a}
& \mbox{definition of }\pappl \\
{}= \pair{(\comp)\, g}{\apure_{\stream{M}}\, (\comp) \appl \tau_g} \pappl \pair{f}{\tau_f} \pappl \pair{a}{\tau_a}
& \mbox{definition of }\pappl \\
{}= \pair{(\comp)}{\apure_{\stream{M}}\,(\comp)} \pappl \pair{g}{\tau_g} \pappl \pair{f}{\tau_f} \pappl \pair{a}{\tau_a}
& \mbox{definition of }\pappl \\
{}= \ppair{\comp} \pappl \pstr{g} \pappl \pstr{f} \pappl \pstr{a}
\end{array}
$$
\end{proof}


\begin{theorem}\label{lemma:appl_laws}
If $M$ is an applicative functor, $\stream{M}$ is also an applicative functor.
\end{theorem}

\section{Monsters are Not Monads}

\vcomm{Counterexample showing that taking the diagonal as join doesn't work}

\section{Conclusion}

We have studied the (co-)algebraic properties of monadic streams ({\em monsters}).
Specifically, we discovered that they inherit from their base operator $M$ the property of being a functor and an applicative functor, but not necessarily that of being a monad.

In the introductory motivation, we illustrated the use of monsters with several instances: for specific choices of $M$, we obtain useful data structures, like lazy lists, finitely branching trees, interactive processes, state machines.

Concurrently with this theoretical study, we also developed an extensive Haskell library for monsters.
In addition to the concrete realization of the type class instances described here, we also programmed a large collection of functions and operators on monsters.
Some of them are generalizations of analogous definitions for pure lists and streams, others are new and original.
The complete Haskell library is publicly available at this address:
\begin{center}\repourl\end{center}

In the near future we will publish two additional articles:
first, a comprehensive description of the monster library, expounding the practical programming side of monsters, together with several applications;
second, an extension of our categorical study of type classes of monsters, with the specific goal to prove that it is impossible to give a universal monad instance for monsters.


\bibliography{monster_biblio.bib}

%\appendix
%
\section{Appendix}

\hide{
\begin{lemmaa}{\ref{lemma:monster_pure}}
$$
\apure_{\stream{M}}\,a = \mcons\,(\apure_M\,\langle a, \apure_{\stream{M}}\,a\rangle)
$$
\end{lemmaa}
\begin{proof}
By unfolding definitions and using the identity law for the applicative functor $M$:
$$
\begin{array}{ll}
\apure_{\stream{M}}\,a \\
{}= a \acons \apure_{\stream{M}}\,a
  & \mbox{definition of }\apure_{\stream{M}} \\
{}= (\apure_M\,a) \fcons (\apure_{\stream{M}}\,a)
  & \mbox{definition of }\acons\\
{}=  \mcons\, (M\,(\lambda a. \langle a, \apure_{\stream{M}}\,a\rangle)\,(\apure_M\,a))
  & \mbox{definition of }\fcons\\
{}= \mcons\, (\apure_M\,(\lambda a. \langle a, \apure_{\stream{M}}\,a\rangle) \appl \apure_M\,a)
  & \mbox{functoriality of applicatives}\\
{}= \mcons\, (\apure_M\,((\lambda a. \langle a, \apure_{\stream{M}}\,a\rangle)\,a))
  & \mbox{the homomorphism law for }M\\
{}= \mcons\, (\apure_M\,\langle a, \apure_{\stream{M}}\,a\rangle)
  & \mbox{function application}
\end{array}
$$
\end{proof}
}
%---

\hide{
\begin{lemmaa}{\ref{lemma:pure_lift}}
For an applicative functor $M$, with $a:A$, $b:B$ and $(\oplus) : A \rightarrow B \rightarrow C$:
$$
(\apure_M\, a) \alift{\oplus} (\apure_M\, b) = \apure_M\, (a \oplus b)
$$
\end{lemmaa}
\begin{proof}
$$
\begin{array}{ll}
(\apure_M\, a) \alift{\oplus} (\apure_M\, b) \\
{}= \apure_M\, (\oplus) \appl \apure_M\, a \appl \apure_M\, b
  & \mbox{definition of }\alift{-}\\
{}= \apure_M\, (a\, \oplus) \appl \apure_M\, b
  & \mbox{homomorphism law of }M\\
{}= \apure_M\, (a \oplus b)
  & \mbox{homomorphism law of }M\\
\end{array}
$$
\end{proof}

%---

\begin{lemmaa}{\ref{lemma:applicative_flip}}
Assuming $M$ is an applicative, with $f : A \rightarrow B \rightarrow C$, $m_a : M\, A$ and $b : B$:
$$
\apure_M\, f \appl m_a \appl \apure_M\, b = \apure_M\, (\lambda a . f\, a\, b) \appl m_a
$$
\end{lemmaa}
\begin{proof}
Since the operator $\appl$ is left-associative, the left hand side of the equality is read as:
$$
(\apure_M\, f \appl m_a) \appl \apure_M\, b
$$
which means that we can apply the interchange law to it.
In the application of the law we abstract a variable $h$ of type $B\rightarrow C$:
$$
\begin{array}{ll}
\apure_M\, f \appl m_a \appl \apure_M\, b \\
{} =  \apure_M\, (\lambda h. h\, b) \appl (\apure_M\, f \appl m_a)
 & \mbox{interchange law of }M\\
{} =  (\apure_M\, (\lambda h. h\, b) \alift{\comp} \apure_M\, f) \appl m_a
 & \mbox{composition law of }M\\
{} =  \apure_M ( (\lambda h. h\, b) \comp f ) ) \appl m_a
 & \mbox{Lemma \ref{lemma:pure_lift}}\\
{} =  \apure_M (\lambda a. f\, a\, b ) \appl m_a
 & \mbox{definition of composition}\\
\end{array}
$$
\end{proof}
}
%---

\begin{lemmaa}{\ref{lemma:pappl_comp_appl}}
Let $m_g:\mstream{B\rightarrow C}$, $m_f:\mstream{A\rightarrow B}$, $m_a:\mstream{A}$, then:
$$
m_g \alift{\pappl} (m_f \alift{\pappl} m_a)
 = \apure\, (\lambda \pstr{g}. \lambda \pstr{f}. \lambda \pstr{a}. \pstr{g} \pappl (\pstr{f}\pappl \pstr{a})) \appl m_g \appl m_f \appl m_a
$$
\end{lemmaa}
\begin{proof}
$$
\begin{array}{ll}
m_g \alift{\pappl} (m_f \alift{\pappl} m_a) \\
{}= \apure\, (\pappl) \appl m_g \appl (\apure\, (\pappl) \appl m_f \appl m_a) \\
 \mbox{definition of }\alift{-}\\
{}= (\apure\, (\pappl) \appl m_g) \alift{\comp} (\apure\, (\pappl) \appl m_f)  \appl m_a \\
 \mbox{composition law for }M \\
{}=   \apure\, (\comp) \appl  (\apure\, (\pappl) \appl m_g) \appl (\apure\, (\pappl) \appl m_f) \appl m_a \\
 \mbox{definition of }\alift{-}\\
{}=   (\apure\, (\comp) \alift{\comp}  \apure\, (\pappl)) \appl m_g \appl (\apure\, (\pappl) \appl m_f) \appl m_a \\
 \mbox{composition law for }M \\
{}=   ( \apure\, ((\comp) \comp (\pappl)) \appl m_g ) \appl (\apure\, (\pappl) \appl m_f)   \appl m_a\\
 \mbox{Lemma \ref{lemma:pure_lift}}\\
{}=   ( \apure\, (\clift{\pappl}{\pstream{A}}) \appl m_g ) \appl (\apure\, (\pappl) \appl m_f)   \appl m_a\\
 \mbox{definition}\\
{}=   ( ( \apure\, (\clift{\pappl}{\pstream{A}})  \appl m_g ) \alift{\comp} \apure\, (\pappl)) \appl m_f  \appl m_a\\
 \mbox{composition law for }M \\
{}=  \apure\, (\comp) \appl (\apure\, (\clift{\pappl}{\pstream{A}})  \appl m_g ) \appl \apure\, (\pappl)  \appl m_f  \appl m_a\\
 \mbox{definition of }\alift{-} \\
{}=   (\apure\, (\comp) \alift{\comp}  \apure\, (\clift{\pappl}{\pstream{A}}) ) \appl m_g \appl \apure\, (\pappl) \appl m_f \appl m_a\\
 \mbox{composition law for }M \\
{}=   \apure\, ((\comp) \comp \clift{\pappl}{\pstream{A}}) \appl m_g \appl \apure\, (\pappl) \appl m_f \appl m_a\\
 \mbox{Lemma \ref{lemma:pure_lift}}\\
{}=    \apure\, (\clift{\clift{\pappl}{\pstream{A}}}{\pstream{A\rightarrow B}}) \appl m_g \appl \apure\, (\pappl) \appl m_f  \appl m_a\\
 \mbox{definition}\\
{}= \apure\, (\lambda \pstr{g} . \pstr{g} \clift{\clift{\pappl}{\pstream{A}}}{\pstream{A\rightarrow B}} (\pappl) ) \appl m_g  \appl m_f  \appl m_a\\
 \mbox{Lemma \ref{lemma:applicative_flip}}\\
{}=  \apure\, (\lambda \pstr{g} . \lambda \pstr{f}. \lambda \pstr{a} . (\pstr{g}\, \pappl (\pstr{f} \pappl \pstr{a})) ) \appl m_g  \appl m_f  \appl m_a\\
 \mbox{Lemma \ref{lemma:comp_comp_comp}}\\
\end{array}
$$
\end{proof}

% ---
\newpage

\begin{lemmaa}{\ref{lemma:ppair}}
$$
(\apure_M\,\ppair{\comp})\, \alift{\pappl}\, m_g\, \alift{\pappl}\, m_f\, \alift{\pappl}\, m_a
= \apure_M\,(\lambda \pstr{g}. \lambda \pstr{f}. \lambda \pstr{a}. \ppair{\comp}\, \pappl\, \pstr{g} \, \pappl\, \pstr{f} \, \pappl\, \pstr{a})\, \appl\, m_g\, \appl\, m_f\, \appl\, m_a 
$$
\end{lemmaa}
\begin{proof}
$$
\begin{array}{ll}
(\apure_M\,\ppair{\comp}) \alift{\pappl} m_g \alift{\pappl} m_f \alift{\pappl} m_a\\
{}=
(\apure_M\,(\pappl) \appl \apure_M\,\ppair{\comp} \appl m_g) \alift{\pappl} m_f \alift{\pappl} m_a\\
\mbox{definition of }\alift{-}\\
{}= (\apure_M\,(\ppair{\comp}\, \pappl) \appl m_g) \alift{\pappl} m_f \alift{\pappl} m_a\\
 \mbox{homomorphism law of }M\\
{}= (\apure_M\,(\pappl) \appl (\apure_M\,(\ppair{\comp}\, \pappl) \appl m_g) \appl m_f) \alift{\pappl} m_a\\
\mbox{definition of }\alift{-}\\
{}= (
       (\apure_M\,(\pappl) \alift{\comp} \apure_M\,(\ppair{\comp} \pappl {}))
       \appl m_g
     \appl m_f) \alift{\pappl} m_a\\
\mbox{composition law of }M\\
{}= ((
       \apure_M\,((\pappl) \comp (\ppair{\comp} \pappl {}))
       \appl m_g
     ) \appl m_f) \alift{\pappl} m_a\\
\mbox{Lemma \ref{lemma:pure_lift}}\\
{}= \apure_M\,(\pappl) \appl
    ((
       \apure_M\,((\pappl) \comp (\ppair{\comp} \pappl {}))
       \appl m_g
     ) \appl m_f) \appl m_a\\
\mbox{definition of }\alift{-}\\
{}= (\apure_M\,(\pappl) \alift{\comp}
     (\apure_M\,((\pappl) \comp (\ppair{\comp} \pappl {}))
       \appl m_g))
    \appl m_f \appl m_a\\
\mbox{composition law of }M\\
{}= \apure_M\,(\comp) \appl
     \apure_M\,(\pappl) \appl
     (\apure_M\,((\pappl) \comp (\ppair{\comp} \pappl {}))
       \appl m_g)
    \appl m_f \appl m_a\\
\mbox{definition of }\alift{-}\\
{}= \apure_M\,((\pappl) \comp {}) \appl
     (\apure_M\,((\pappl) \comp (\ppair{\comp} \pappl {}))
       \appl m_g)
    \appl m_f \appl m_a\\
 \mbox{homomorphism law of }M\\
{}= (\apure_M\,((\pappl) \comp {}) \alift{\comp}
     \apure_M\,((\pappl) \comp (\ppair{\comp} \pappl {})))
    \appl m_g \appl m_f \appl m_a\\
\mbox{composition law of }M\\
{}= \apure_M\,(((\pappl) \comp {}) \comp
                ((\pappl) \comp (\ppair{\comp} \pappl {})))
    \appl m_g \appl m_f \appl m_a\\
\mbox{Lemma \ref{lemma:pure_lift}}
\end{array}
$$

We are left with the daunting high-order expression:
$$
((\pappl) \comp {}) \comp ((\pappl) \comp (\ppair{\comp} \pappl {}))
$$
where the composition operator $\comp$ is used in several places with different types.
The whole expression has type $\pstream{B\rightarrow C}\rightarrow \pstream{A\rightarrow B} \rightarrow \pstream{A}\rightarrow \pstream{C}$.
If we apply it to elements $\pstr{g}$, $\pstr{f}$, $\pstr{a}$ and simplify by repeatedly applying the definition of composition, we get:
$$
\begin{array}{l}
(((\pappl) \comp {}) \comp ((\pappl) \comp (\ppair{\comp} \pappl {})))\,
\pstr{g}\,\pstr{f}\,\pstr{a}\\
{}= (((\pappl) \comp {}) (((\pappl) \comp (\ppair{\comp} \pappl {}))\,\pstr{g}))\,
    \pstr{f}\,\pstr{a}\\
{}= (((\pappl) \comp {}) ((\pappl) (\ppair{\comp} \pappl\pstr{g})))\,
    \pstr{f}\,\pstr{a}\\
{}= ((\pappl) ((\ppair{\comp} \pappl\pstr{g}) \pappl \pstr{f}))\,\pstr{a}\\
{}= (\ppair{\comp} \pappl\pstr{g}) \pappl \pstr{f})\pappl\pstr{a}\\
\end{array}
$$
Therefore:
$$
((\pappl) \comp {}) \comp ((\pappl) \comp (\ppair{\comp} \pappl {}))
= \lambda \pstr{g}. \lambda \pstr{f}. \lambda \pstr{a}.
  \ppair{\comp} \pappl\pstr{g} \pappl \pstr{f} \pappl\pstr{a}
$$
which concludes the proof.
\end{proof}

%---

\begin{lemmaa}{\ref{lemma:general_bind_law}}
For $a : A$, $f : A \rightarrow M\, B$ and assuming $M$ is a monad:
$$
(\join\, \comp\,  M\, f\, \comp\, \return)\, n = \mcons\, (M\,(\lambda a . \pair{a}{\join\, (\apure_{\stream{M}}\, (\mathsf{traverse}\, (f\, n)))})\, (\head\, (f\, n)))
$$
\end{lemmaa}

\begin{proof}

$$
\begin{array}{ll}
(\join \comp  M\, f \comp \return)\, n \\
{}= (\join \comp  M\, f)\, (\apure_{\stream{M}}\, n)
  & \mbox{definition of } \return \mbox{ as } \apure \\
{}= \join\, (\apure_{\stream{M}}\, (f \,n))
  & \mbox{the homomorphism law for } \stream{M}\\
{}= \mcons\, ((\uncons\, (\apure_{\stream{M}}\, (f\, n)))\, \bind \\\qquad \,(\lambda \pair{hs}{ts} . M\,(\lambda a . \pair{a}{(\join \comp \stream{M}\,\mathsf{traverse})\, ts})\, (\head\, hs)))
  & \mbox{definition of } \join \\
{}= \mcons\,((\apure_{M}\,\pair{f\, n}{\apure_{\stream{M}}\,(f\, n)})\, \bind \\\qquad \,(\lambda \pair{hs}{ts} . M\,(\lambda a . \pair{a}{(\join \comp \stream{M}\, \mathsf{traverse})\, ts})\, (\head\, hs)))
  & \mbox{definition of } \uncons \mbox{ and Lemma \ref{lemma:monster_pure}}\\
{}= \mcons\, ((\lambda \pair{hs}{ts} . M\,(\lambda a . \pair{a}{(\join \comp \stream{M}\,\mathsf{traverse})\, ts})\, (\head\, hs)) \, \\\qquad (\pair{f\, n}{\apure_{\stream{M}}\,(f\, n)}))
  & \mbox{left identity law of } M \\
{}= \mcons\,(M\,(\lambda a . \pair{a}{(\join \comp \stream{M}\,\mathsf{traverse})\, (\apure_{\stream{M}}\,(f\, n))})\, \\ \qquad (\head\, (f\, n)))
 & \mbox{function application} \\
 {}= \mcons\, (M\,(\lambda a . \pair{a}{\join\, (\apure_{\stream{M}}\, (\mathsf{traverse}\, (f\, n)))})\, (\head\, (f\, n)))
 & \mbox{the homomorphism law for } \stream{M}\\

\end{array} 
$$
\end{proof}

%---

\begin{lemmaa}{\ref{lemma:traverse_fromstepstr}}
$$
\mathsf{traverse}\, (\fromstepstr\, n) = \mcons\, (\lambda s . \pair{\pair{s+n}{\fromstepstr\, (n + 2)}}{s + 2n + 1})
$$
\end{lemmaa} 

\begin{proof}
We prove this by expanding definitions

$$
\begin{array}{ll}
\mathsf{traverse}\, (\fromstepstr\, n) \\
{}= \mathsf{traverse}\, (\mcons\, (\lambda s. \pair{\pair{s}{\fromstepstr\, (n + 1)}}{s + n}))
	& \mbox{definition of }\fromstepstr \\
{}= (\mathsf{absorb} \comp \tail \comp \uncons)\, \\\qquad (\mcons\, (\lambda s. \pair{\pair{s}{\fromstepstr (n + 1)}}{s + n}))
	& \mbox{definition of }\mathsf{traverse} \\
{}= (\mathsf{absorb} \comp \tail)\, (\lambda s. \pair{\pair{s}{\fromstepstr\, (n + 1)}}{s + n})
	& \mbox{definition of }\uncons \\
{}= \mathsf{absorb}\, (\lambda s. \pair{\fromstepstr\, (n + 1)}{s + n})
	& \mbox{definition of }\tail \\
{}= \mathsf{absorb}\, (\lambda s. \pair{(\mcons\, (\lambda s. \pair{\pair{s}{\fromstepstr\, (n + 2)}}{s + (n + 1)}))}{s + n})
	& \mbox{definition of }\fromstepstr \\
{}= (\mcons \comp \join_{\state_{\nat}} \comp \state_{\nat}\,\uncons) \\\qquad \, (\lambda s. \pair{(\mcons\, (\lambda s. \pair{\pair{s}{\fromstepstr\, (n + 2)}}{s + (n + 1)}))}{s + n})
	& \mbox{definition of }\mathsf{absorb} \\
{}= (\mcons \comp \join_{\state_{\nat}})\, \\\qquad (\lambda s. \pair{(\lambda s. \pair{\pair{s}{\fromstepstr\, (n + 2)}}{s + (n + 1)})}{s + n})
	& \mbox{functoriality of }\state_{\nat} \\
{}= \mcons\, (\lambda s. \pair{\pair{s + n}{\fromstepstr\,(n + 2)}}{(s + n) + (n + 1)})
	& \mbox{definition of } \join_{\state_{\nat}} \\
{}= \mcons\, (\lambda s. \pair{\pair{s + n}{\fromstepstr\,(n + 2)}}{s + 2n + 1})
\end{array}
$$
\end{proof}



\end{document}
