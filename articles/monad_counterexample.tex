\section{Monsters are Not Monads}\label{sec:monad}

The pure stream constructor $\strsym$ is known to be a monad, by making the $\join$ method return the diagonal of a stream of streams.
Pure streams are $\Id$-monsters, so we may think it possible to generalize this definition to monsters based on any monad.
However, we show in this session that this fails, by giving a counterexample for state-monsters.

Specific monads may result in monadic monsters, either by using the diagonal as $\join$ or with a different definitions.
For example, $\maybe$-monsters (lazy lists) are a monad by using concatenation as $\join$.

We conjecture that a universal definition (parametric on the base monad) is not possible.
Our discussion of the counterexample for state-monster gives some intuitive reason why it seems unlikely that a monad instantiation exists.
A full proof of impossibility is still an open problem.

A monad is defined as a functor $M$ with a $\return$ function injecting pure elements into it and either a join or a bind operation sequencing monadic actions, satisfying three laws.

$$
\begin{array}{l}
\return : A \rightarrow M\, A \\
\join : M\,(M\, A) \rightarrow M\, A\\
\mbox{ or }
(\bind) : M\, A \rightarrow (A \rightarrow M\, B) \rightarrow M\, B
\end{array}
$$

The $(\bind)$ and $\join$ operations can be defined in terms of each other:

$$
\begin{array}{l}
(\bind\, f) = \join \comp M f \\
\join = (\bind\,\id)
\end{array}
$$

The laws can be expressed either in terms of $\join$ or $(\bind)$.
The following is the formulation in terms of $(\bind)$.
Let $a:A$, $m_a : M\, A$,
$f : A \rightarrow M\, B$, $g : B \rightarrow M\,C$:

$$
\begin{array}{l@{\qquad}l}
(\return\, a)\, \bind \, f = f\, a
  & \mbox{Left identity} \\
m_a\, \bind \, \return = m_a
  & \mbox{Right identity} \\
(m_a \, \bind \, f)\, \bind \, g = m_a\, \bind\, (\lambda a.  f\, a \, \bind \, g)
  & \mbox{Composition} \\
\end{array}
$$

A monad is automatically an applicative functor:
$\apure$ is the same as $\return$ and $\appl$ can be defined as:

$$
m_f\appl m_a = m_f \bind (\lambda f. m_a \bind (\lambda a. f\,a))
$$

The idea of the implementation of the $\join$ operation is that,
given a {\em monster matrix} $\mu:\stream{M}\,(\stream{M}\,A)$, we travel down the diagonal and pick elements in positions $(0,0)$, $(1,1)$, $(2,2)$, etc.
This seems the most sensible approach as it generalizes canonical $\join$ operation of the pure stream monad.
However, since these are monadic streams, accessing those elements requires executing all the monadic actions leading to them.

Let us represent $\mu$ intuitively as follows, denoting the monadic actions by arrow:

$$
\begin{array}[t]{rll}
\mu \larrow{m_0} 
 & \sigma_0 
 & \larrow{m_{00}} a_{00} \larrow{m_{01}} a_{01} \larrow{m_{02}} a_{02} \larrow{} \cdots\\[5pt]
\larrow{m_1} 
 & \sigma_1 
 & \larrow{m_{10}} a_{10} \larrow{m_{11}} a_{11} \larrow{m_{12}} a_{12} \larrow{} \cdots\\[5pt]
\larrow{m_2} 
 & \sigma_2
 & \larrow{m_{20}} a_{20} \larrow{m_{21}} a_{21} \larrow{m_{22}} a_{22} \larrow{} \cdots\\[5pt]
\larrow{} & \cdots
\end{array}
$$

The result of $(\join\,\mu)$ should be the monster $a_{00} \larrow{} a_{11}\larrow{} a_{22} \larrow{} \cdots$.
However, to access these elements, we are forced to execute all the actions leading to them:

$$
\begin{array}{ll}
\join\,\mu 
& \larrow{m_0}\larrow{m_{00}} a_{00}\\[5pt]
& \larrow{m_1}\larrow{m_{10}}\larrow{m_{11}} a_{11}\\[5pt]
& \larrow{m_2}\larrow{m_{20}}\larrow{m_{21}}\larrow{m_{22}} a_{22}\\[5pt]
& \larrow{} \cdots
\end{array}
$$

In order to define it formally, let us introduce some notation.
We use $\unc{\sigma}$ to denote the monadic action associated to the monster $\sigma$, that is $ \unc{\sigma} = \uncons\,\sigma $, just stripping the $\mcons$ constructor away; so $\unc{\mu}$ is the action $m_0$ above.
The $\head$ and $\tail$ functions extract the components of a monster, but of course they can only return a value inside the monad $M$.

$$
\begin{array}{l}
\uncons : \stream{M}\,A \rightarrow M\,(A \times \stream{M}\,A)\\
\uncons\, (\mcons\, m_a) = m_a \\
\,\\
\unc{-} : \stream{M}\,A \rightarrow M\,(A\times \stream{M}\,A)\\
\unc{\sigma} = \uncons\,\sigma
\end{array}
\qquad
\begin{array}{l}
\head : \stream{M}\,A \rightarrow M\,A\\
\head = M\,\fst \comp \uncons\\
\,\\
\tail : \stream{M}\,A \rightarrow M\,(\stream{M}\,A)\\
\tail = M\,\snd \comp \uncons\\
\end{array}
$$

When we traverse a monster to get to its elements, we have to use monadic actions, so we don't get pure elements.
That's the case for $\head$ and $\tail$ above.
However, a monadic action can always be absorbed inside a monster, but using the $\join$ operation for $M$.
This allows us to define a ``pure'' version of the $\tail$ function, which we call $\traverse$:
We first define four auxiliary functions:

$$
\begin{array}{l}
\absorb : M\,(\stream{M}\,A) \rightarrow \stream{M}\,A \\
\absorb =  \mcons \comp \join_{M} \comp M\,\uncons
\end{array}
  \qquad
\begin{array}{l}
\traverse : \stream{M}\,A \rightarrow \stream{M}\,A \\
\traverse = \absorb \comp \tail \comp \uncons
\end{array}
$$

We are ready to give the definition of $\join$:
$$
\begin{array}{ll}
\multicolumn{2}{l}{\join: \stream{M}\,(\stream{M}\,A) \rightarrow \stream{M}\,A}\\
\join\,\mu = \mcons\,(
  & \unc{\mu} \bind
    \lambda \pair{\sigma_a}{\mu'}.\\
  &  (\head\,\sigma_a) \bind \lambda a.\\
  & \return\,\pair{a}{\join\,(\stream{M}\,\traverse\, \mu')} \; )
\end{array}
$$

There is some degree of unpacking, but ultimately, given a monster matrix $\mu$, we get its components, the first row $\sigma_a$ and the rest of the matrix $\mu'$; 
we extract the first element $a$ from $\sigma_a$;
we map $\traverse$ onto $\mu'$ to delete the first column, then we corecursively apply $\join$.

Keeping in mind that the functorial mapping for a monad is defined as $M\,f\,m_a = m_a \bind \lambda a. \return\,(f\,a)$ and rearranging the expression a bit, we have this equivalent one-line characterization:

$$
\join\, \mu = \mcons\, ((\uncons\, \mu) \, \bind \,(\lambda \pair{\sigma_a}{\mu'} . M\,(\lambda a . \pair{a}{(\join\, \comp\, \stream{M}\,\traverse)\, \mu'})\, (\head\, \sigma_a)))
$$

We next show that this definition of $\join$ fails to satisfy the monad laws.
We give a counterexample of the left identity law.
First of all, let us translate the law from the $\bind$ formulation to the equivalent $\join$ version for monsters, where $a:A$ and $f:A\rightarrow \stream{M}\,B$:

$$
\begin{array}{l}
(\return\, a)\, \bind \, f = f\, a\\
(\join \comp \stream{M}\,f)\,(\return\,a) = f\,a
\end{array}
$$

So the law says that $\join \comp (\stream{M}\,f)\comp \return = f$.

Intuitively, $(f\,a)$ is an $M$-monster of $B$ values:

$$
f\,a \larrow{m_0} b_0 \larrow{m_1} b_1 \larrow{m_2} b_2 \larrow{}\cdots
$$

On the left-hand side, $\return\,a$ is a pure streams of repetitions of $a$; when we map $f$ on it, we get a matrix in which every row is equal to $f\,a$:

$$
\begin{array}[t]{rll}
(\stream{M}\,f)\,(\return\,a) \larrow{} 
 & f\,a 
 & \larrow{m_0} b_0 \larrow{m_1} b_1 \larrow{m_2} b_2 \larrow{}\cdots\\[5pt]
\larrow{} 
 & f\,a 
 & \larrow{m_0} b_0 \larrow{m_1} b_1 \larrow{m_2} b_2 \larrow{}\cdots\\[5pt]
\larrow{} 
 & f\,a 
 & \larrow{m_0} b_0 \larrow{m_1} b_1 \larrow{m_2} b_2 \larrow{}\cdots\\[5pt]
\larrow{} & \cdots
\end{array}
$$

The $\join$ operation then travels down the diagonal of this matrix:

$$
\begin{array}{ll}
\join\,((\stream{M}\,f)\,(\return\,a))
& \larrow{m_0} b_0\\[5pt]
& \larrow{m_0}\larrow{m_1} b_1\\[5pt]
& \larrow{m_0}\larrow{m_1}\larrow{m_2} b_2\\[5pt]
& \larrow{} \cdots
\end{array}
$$

We can see here why we don't necessarily obtain the same result as $(f\,a)$: all the initial monadic actions at each row need to be executed every time to get to the element on the diagonal at that depth.

With this idea in mind, we give a formal counterexample with state machines.
Recall the notation for state-monsters: $\stmon{S} = \stream{\state_S}$, where $\state_S A = S \rightarrow A \times S$, representing the $\state$-transformer functor with states of type $S$ and return values of type $A$. 
We assume it is equipped with the usual monad operations for the $\state$ monad. We call $\state$-monsters  {\em state machines}.

$$
\begin{array}{l}
\fromstepstr : \nat \rightarrow \stmon{\nat}\,\nat \\
\fromstepstr\, n = \mcons\, (\lambda s. \pair{\pair{s}{\fromstepstr (n + 1)}}{s + n})
\end{array}
$$

This function generates a state machine which, when given a number, generates a stream where the first two numbers differ by $n$, the next two differ by $n + 1$, and so on.
% Runner function
%To observe the results of monsters of this type, we need a `runner' function that generates a pure stream:
%$$
%\begin{array}{l}
%\strun :: \stmon{S}\, A \rightarrow S \rightarrow \stream{\Id}\,A \\
%\strun\, m_0\, s_0 = n \acons (\strun\, m_1\, %s_1) \\
%\qquad \where\, \pair{\pair{n}{m_1}}{s_1} = (\uncons\, m_0)\, s_0
%\end{array}
%$$

You can `run' this state machine by threading the state through each nested function, and collecting the results. This requires an initial seed state. For example:

$$
\strun\,(\fromstepstr\, 1)\, 2 = 2\, \acons\, 3\, \acons\, 5\, \acons\, 8\, \acons\, 12\, \acons\, \dots
$$

We now show that this specific case of the left identity monad law does not hold. 

$$
(\join \comp \stmon{\nat}\, \fromstepstr \comp \return)\, n 
\neq \fromstepstr\, n
$$

The law is expanded using the definition of $\bind$ in terms of $\join$. 
To show that the equality does not hold, the following two lemmas are useful to simplify reasoning. Any uses of $\join$ without a subscript refer to the $\join$ operation for monsters, otherwise the subscript will indicate the monad.

\begin{lemma}\label{lemma:general_bind_law}
For $a : A$, $f : A \rightarrow M\, B$ and assuming $M$ is a monad:

$$
(\join\, \comp\,  M\, f\, \comp\, \return)\, n = \mcons\, (M\,(\lambda a . \pair{a}{\join\, (\apure_{\stream{M}}\, (\traverse\, (f\, n)))})\, (\head\, (f\, n)))
$$
\end{lemma}

\begin{lemma}\label{lemma:traverse_fromstepstr}
$$
\traverse\, (\fromstepstr\, n) = \mcons\, (\lambda s . \pair{\pair{s+n}{\fromstepstr\, (n + 2)}}{s + 2n + 1})
$$
\end{lemma} 

The proofs of these lemmas are mostly simple cases of expanding definitions, and are included in the appendix.

\begin{theorem}\label{lemma:monad_counterexample}
Monadic streams equipped with a `diagonal' $\join$ operation are not monads.
\end{theorem}
\begin{proof}
We use $tts = \join\, (\traverse\, (\traverse\, (\fromstepstr\, n)))$ to simplify the following reasoning, as we only need to unfold the definitions so far in order to demonstrate the conjecture. \\

First we expand $\fromstepstr\, n$:
$$
\begin{array}{ll}
\fromstepstr\, n \\
{}= \mcons\, (\lambda s. \pair{\pair{s}{\fromstepstr\, (n + 1)}}{s + n})\\
{}= \mcons\, (\lambda s. \pair{\pair{s}{(\mcons\, (\lambda s. \pair{\pair{s}{\fromstepstr\, (n + 2)}}{s + n + 1}))}}{s + n}) \\
\end{array} 
$$
We then compare it to the other side of the left-identity law
$$
\begin{array}{l}
(\join \comp  \stmon{\nat}\, \fromstepstr \comp \return)\, n \\
{}= \mcons\, (\state_{\nat} \,(\lambda a . \pair{a}{\join\, (\apure_{\stmon{\nat}}\, (\traverse\, (\fromstepstr\, n)))})\, (\head\, (\fromstepstr\, n))) \\
 \qquad\mbox{Lemma \ref{lemma:general_bind_law}}  \\
{}= \mcons\, (\state_{\nat}\,(\lambda a . \pair{a}{\join\, (\apure_{\stmon{\nat}}\,(\traverse\, (\fromstepstr\, n)))})\, (\lambda s. \pair{s}{s + n})) \\
 \qquad\mbox{definition of } \head \mbox{ and } \fromstepstr \\
{}= \mcons\, (\state_{\nat}\,(\lambda a . \pair{a}{  
\mcons\, (\state_{\nat} \,(\lambda a . \pair{a}{tts})\, (\head\, (\traverse\, (\fromstepstr\, n))))
})\, (\lambda s. \pair{s}{s + n})) \\
 \qquad\mbox{Lemma \ref{lemma:general_bind_law}}\\
{}= \mcons\, (\state_{\nat}\,(\lambda a . \pair{a}{  
\mcons\, (\state_{\nat} \,(\lambda a . \pair{a}{tts})\, \\\qquad(\head\, (\mcons\, (\lambda s . \pair{\pair{s+n}{\fromstepstr\, (n + 2)}}{s + 2n + 1}))))
})\, (\lambda s. \pair{s}{s + n})) \\
 \qquad\mbox{Lemma \ref{lemma:traverse_fromstepstr}}\\
{}=  \mcons\, (\state_{\nat}\,(\lambda a . \pair{a}{  
\mcons\, (\state_{\nat} \,(\lambda a . \pair{a}{tts})\, (\lambda s . \pair{s + n}{s + 2n + 1}))
})\, (\lambda s. \pair{s}{s + n}))
\\ \qquad\mbox{definition of } \head \\
{}= \mcons\, (\state_{\nat}\,(\lambda a . \pair{a}{  
\mcons\, (\lambda s . \pair{\pair{s + n}{tts}}{s + 2n + 1})
})\, (\lambda s. \pair{s}{s + n})) \\
\qquad\mbox{functor definition of } \state_{\nat}  \\
{}= \mcons\, (\lambda s. \pair{\pair{s}{\mcons\, (\lambda s . \pair{\pair{s + n}{tts}}{s + 2n + 1})}}{s + n}) \\ 
\qquad\mbox{functor definition of } \state_{\nat} \\
\end{array} 
$$
It is clear by inspection that 
$$
\begin{array}{l}
\mcons\, (\lambda s. \pair{\pair{s}{(\mcons\, (\lambda s. \pair{\pair{s}{\fromstepstr\, (n + 2)}}{s + n + 1}))}}{s + n}) \neq \\
\qquad \mcons\, (\lambda s. \pair{\pair{s}{\mcons\, (\lambda s . \pair{\pair{s + n}{tts}}{s + 2n + 1})}}{s + n})
\end{array} 
$$
without expanding $tts$ and $\fromstepstr\, (n + 2)$ any further.
\end{proof}

The issue is that the first monadic action (that of $\fromstepstr\, n$) is repeated twice in the monster resulting from the $\join$ operation. 
This indicates that monsters may still be monads if the underlying functor is an idempotent monad instead. 
We have yet to prove this, and we are also looking for a weaker condition on the underlying functor, as idempotency is a strong requirement. \\

An argument could be made that the only possible definition of $\join$ is the one we consider here. Pure streams are simply monsters with the underlying monad $\Id$, and taking the diagonal is the canonical $\join$ operation in this case. By naturality of the $\join$ operation, the definition may have to be independent of the particular underlying monad chosen. If this is the case, then monsters are not monads, because as demonstrated, the diagonal $\join$ operation does not satisfy the required laws for $\stmon{\nat}$, so either a different definition is required (which could violate naturality), or monsters (with a monadic underlying functor) are not monads in general. We suspect it is the latter, and one goal of our future work is to verify this.
