\section{Monsters are Not Monads}

We show here that the seemingly most natural way of defining a $\join$ operation on monsters doesn't satisfy the monad laws. \\

A monad can be defined in bind form as a functor $M$, equipped with two methods, $\return$ and $\bind$:
$$
\begin{array}{l}
\return : A \rightarrow M\, A \\
(\bind) : M\, A \rightarrow (A \rightarrow M\, B) \rightarrow M\, B
\end{array}
$$
If a functor is a monad, it is an applicative functor. The correct definition for the $\return$ operation is always the same as that of $\apure$, and visa-versa. \\

For a functor $M$ the monad laws are as follows, where $a:A$, $m_a : M\, A$
$f : A \rightarrow M\, B$, $g : B \rightarrow M\,C$. We write these in terms of $\bind$, derived from Kleisli arrow composition and identity laws:
$$
\begin{array}{l@{\qquad}l}
(\return\, a)\, \bind \, f = f\, a
  & \mbox{Left identity} \\
m_a\, \bind \, \return = m_a
  & \mbox{Right identity} \\
(m_a \, \bind \, f)\, \bind \, g = m_a\, \bind\, (\lambda a.  f\, a \, \bind \, g)
  & \mbox{Composition} \\
\end{array}
$$
A more categorical way of defining a monad is in terms $\join : M\,(M\, A) \rightarrow M\, A$.
$$
\join : M\,(M\, A) \rightarrow M\, A
$$
The $\bind$ operation can be defined in terms of $\join$, and visa-versa.
$$
\begin{array}{l}
(\bind\, f) = \join \comp M f \\
\join = (\bind\,\id)
\end{array}
$$

By showing that a natural definition of $\join$ does not satisfy the monad laws, we also show that the corresponding definition of $\bind$ does not either.

Our naive implementation of $\join$ takes the `diagonal' to flatten a monster of monsters. This means taking the first element of the first monster, the second element of the second monster, and so on. This was - to us - the most natural way to approach this definition initially, as it is similar to the canonical $\join$ operation of the pure stream monad. \\

We first define three auxiliary functions:
$$
\begin{array}{l}
\uncons : \stream{M}\,A \rightarrow M\,(A \times \stream{M}\,A)\\
\uncons\, (\mcons\, m_a) = m_a \\
\,\\
\mathsf{absorb} : M\,(\stream{M}\,A) \rightarrow \stream{M}\,A \\
\mathsf{absorb} =  \mcons \comp \join_{M} \comp M\,\uncons \\
\,\\
\mathsf{traverse} : \stream{M}\,A \rightarrow \stream{M}\,A \\
\mathsf{traverse} = \mathsf{absorb} \comp \tail \comp \uncons
\end{array}
$$

Our definition of $\join$ is as follows, where $s, ts : \stream{M}\,(\stream{M}\, A)$, 
$hs : \stream{M}\, A$, $a : M\, A$. We assume that the type operator $M$ is equipped with $\return_M$ and $\join_M$ that satisfy the monad laws.
$$
\join\, s = \mcons\, ((\uncons\, s) \, \bind \,(\lambda \pair{hs}{ts} . M\,(\lambda a . \pair{a}{(\join\, \comp\, \stream{M}\,\mathsf{traverse})\, ts})\, (\head\, hs)))
$$
We introduce the functor $\state\stream{S} = \stream{\state_S}$, where $\state_S A = S \rightarrow A \times S$, representing the $\state$ functor with states of type $S$, and return values of type $A$. We assume it is equipped with the usual monad operations for the $\state$ monad. We call these types of monsters (those with an underlying $\state$ monad) state machines.
$$
\begin{array}{l}
\fromstepstr : \nat \rightarrow \state\stream{\nat}\,\nat \\
\fromstepstr\, n = \mcons\, (\lambda s. \pair{\pair{s}{\fromstepstr (n + 1)}}{s + n})
\end{array}
$$
This function generates a state machine which, when given a number, generates a stream where the first two numbers differ by $n$, the next two differ by $n + 1$, and so on.
% Runner function
%To observe the results of monsters of this type, we need a `runner' function that generates a pure stream:
%$$
%\begin{array}{l}
%\mathsf{run}\state\stream{} :: \state\stream{S}\, A \rightarrow S \rightarrow \stream{\Id}\,A \\
%\mathsf{run}\state\stream{}\, m_0\, s_0 = n \acons (\mathsf{run}\state\stream{}\, m_1\, %s_1) \\
%\qquad \where\, \pair{\pair{n}{m_1}}{s_1} = (\uncons\, m_0)\, s_0
%\end{array}
%$$

You can `run' this state machine by threading the state through each nested function, and collecting the results. This requires an initial seed state. For example:
$$
\mathsf{run}\state\stream{}\,(\fromstepstr\, 1)\, 2 = 2\, \acons\, 3\, \acons\, 5\, \acons\, 8\, \acons\, 12\, \acons\, \dots
$$\\
We now show that this specific case of the left identity monad law does not hold. 
$$
(\join \comp \state\stream{\nat}\, \fromstepstr \comp \return)\, n = \fromstepstr\, n
$$

The law is expanded using the definition of $\bind$ in terms of $\join$. 
To show this does not hold, the following lemmas are useful to simplify reasoning. Any uses of $\join$ without a subscript refer to the $\join$ operation for monsters, otherwise the subscript will indicate the monad.

\begin{lemma}\label{lemma:general_bind_law}
For $a : A$, $f : A \rightarrow M\, B$ and assuming $M$ is a monad:
$$
(\join\, \comp\,  M\, f\, \comp\, \return)\, n = \mcons\, (M\,(\lambda a . \pair{a}{\join\, (\apure_{\stream{M}}\, (\mathsf{traverse}\, (f\, n)))})\, (\head\, (f\, n)))
$$
\end{lemma}

\begin{proof}
$$
\begin{array}{ll}
(\join \comp  M\, f \comp \return)\, n \\
{}= (\join \comp  M\, f)\, (\apure_{\stream{M}}\, n)
  & \mbox{by definition of } \return \mbox{ as } \apure \\
{}= \join\, (\apure_{\stream{M}}\, (f \,n))
  & \mbox{by the homomorphism law for } \stream{M}\\
{}= \mcons\, ((\uncons\, (\apure_{\stream{M}}\, (f\, n)))\, \bind \\\qquad \,(\lambda \pair{hs}{ts} . M\,(\lambda a . \pair{a}{(\join \comp \stream{M}\,\mathsf{traverse})\, ts})\, (\head\, hs)))
  & \mbox{by definition of } \join \\
{}= \mcons\,((\apure_{M}\,\pair{f\, n}{\apure_{\stream{M}}\,(f\, n)})\, \bind \\\qquad \,(\lambda \pair{hs}{ts} . M\,(\lambda a . \pair{a}{(\join \comp \stream{M}\, \mathsf{traverse})\, ts})\, (\head\, hs)))
  & \mbox{by definition of } \uncons \mbox{ and Lemma \ref{lemma:monster_pure}}\\
{}= \mcons\, ((\lambda \pair{hs}{ts} . M\,(\lambda a . \pair{a}{(\join \comp \stream{M}\,\mathsf{traverse})\, ts})\, (\head\, hs)) \, \\\qquad (\pair{f\, n}{\apure_{\stream{M}}\,(f\, n)}))
  & \mbox{by left identity law of } M \\
{}= \mcons\,(M\,(\lambda a . \pair{a}{(\join \comp \stream{M}\,\mathsf{traverse})\, (\apure_{\stream{M}}\,(f\, n))})\, \\ \qquad (\head\, (f\, n)))
 & \mbox{function application} \\
 {}= \mcons\, (M\,(\lambda a . \pair{a}{\join\, (\apure_{\stream{M}}\, (\mathsf{traverse}\, (f\, n)))})\, (\head\, (f\, n)))
 & \mbox{by the homomorphism law for } \stream{M}\\

\end{array} 
$$
\end{proof}


\begin{lemma}\label{lemma:traverse_fromstepstr}
$$
\mathsf{traverse}\, (\fromstepstr\, n) = \mcons\, (\lambda s . \pair{\pair{s+n}{\fromstepstr\, (n + 2)}}{s + 2n + 1})
$$
\end{lemma} 

\begin{proof}
We prove this by expanding definitions
$$
\begin{array}{ll}
\mathsf{traverse}\, (\fromstepstr\, n) \\
{}= \mathsf{traverse}\, (\mcons\, (\lambda s. \pair{\pair{s}{\fromstepstr\, (n + 1)}}{s + n}))
	& \mbox{by definition of }\fromstepstr \\
{}= (\mathsf{absorb} \comp \tail \comp \uncons)\, \\\qquad (\mcons\, (\lambda s. \pair{\pair{s}{\fromstepstr (n + 1)}}{s + n}))
	& \mbox{by definition of }\mathsf{traverse} \\
{}= (\mathsf{absorb} \comp \tail)\, (\lambda s. \pair{\pair{s}{\fromstepstr\, (n + 1)}}{s + n})
	& \mbox{by definition of }\uncons \\
{}= \mathsf{absorb}\, (\lambda s. \pair{\fromstepstr\, (n + 1)}{s + n})
	& \mbox{by definition of }\tail \\
{}= \mathsf{absorb}\, (\lambda s. \pair{(\mcons\, (\lambda s. \pair{\pair{s}{\fromstepstr\, (n + 2)}}{s + (n + 1)}))}{s + n})
	& \mbox{by definition of }\fromstepstr \\
{}= (\mcons \comp \join_{\state_{\nat}} \comp \state_{\nat}\,\uncons) \\\qquad \, (\lambda s. \pair{(\mcons\, (\lambda s. \pair{\pair{s}{\fromstepstr\, (n + 2)}}{s + (n + 1)}))}{s + n})
	& \mbox{by definition of }\mathsf{absorb} \\
{}= (\mcons \comp \join_{\state_{\nat}})\, \\\qquad (\lambda s. \pair{(\lambda s. \pair{\pair{s}{\fromstepstr\, (n + 2)}}{s + (n + 1)})}{s + n})
	& \mbox{by functoriality of }\state_{\nat} \\
{}= \mcons\, (\lambda s. \pair{\pair{s + n}{\fromstepstr\,(n + 2)}}{(s + n) + (n + 1)})
	& \mbox{by definition of } \join_{\state_{\nat}} \\
{}= \mcons\, (\lambda s. \pair{\pair{s + n}{\fromstepstr\,(n + 2)}}{s + 2n + 1})
	& \mbox{non-trivial arithmetic} \\
\end{array}
$$
\end{proof}

\begin{conjecture}\label{lemma:monad_counterexample}
Monadic streams equipped with a `diagonal' $\join$ operation are not monads.
\end{conjecture}
\begin{proof}
We use $tts = \join\, (\mathsf{traverse}\, (\mathsf{traverse}\, (\fromstepstr\, n)))$ to simplify the following reasoning, as we only need to unfold the definitions so far in order to demonstrate the conjecture. \\

First we expand $\fromstepstr\, n$:
$$
\begin{array}{ll}
\fromstepstr\, n \\
{}= \mcons\, (\lambda s. \pair{\pair{s}{\fromstepstr\, (n + 1)}}{s + n})\\
{}= \mcons\, (\lambda s. \pair{\pair{s}{(\mcons\, (\lambda s. \pair{\pair{s}{\fromstepstr\, (n + 2)}}{s + n + 1}))}}{s + n}) \\
\end{array} 
$$
We then compare it to the other side of the left-identity law
$$
\begin{array}{ll}
(\join \comp  \state\stream{\nat}\, \fromstepstr \comp \return)\, n \\
{}= \mcons\, (\state_{\nat} \,(\lambda a . \pair{a}{\join\, (\apure_{\state\stream{\nat}}\, (\mathsf{traverse}\, (\fromstepstr\, n)))})\, (\head\, (\fromstepstr\, n))) \\
 \mbox{by Lemma \ref{lemma:general_bind_law}}  \\
{}= \mcons\, (\state_{\nat}\,(\lambda a . \pair{a}{\join\, (\apure_{\state\stream{\nat}}\,(\mathsf{traverse}\, (\fromstepstr\, n)))})\, (\lambda s. \pair{s}{s + n})) \\
 \mbox{by definition of } \head \mbox{ and } \fromstepstr \\
{}= \mcons\, (\state_{\nat}\,(\lambda a . \pair{a}{  
\mcons\, (\state_{\nat} \,(\lambda a . \pair{a}{tts})\, (\head\, (\mathsf{traverse}\, (\fromstepstr\, n))))
})\, (\lambda s. \pair{s}{s + n})) \\
 \mbox{by Lemma \ref{lemma:general_bind_law}}\\
{}= \mcons\, (\state_{\nat}\,(\lambda a . \pair{a}{  
\mcons\, (\state_{\nat} \,(\lambda a . \pair{a}{tts})\, \\\qquad(\head\, (\mcons\, (\lambda s . \pair{\pair{s+n}{\fromstepstr\, (n + 2)}}{s + 2n + 1}))))
})\, (\lambda s. \pair{s}{s + n})) \\
 \mbox{by Lemma \ref{lemma:traverse_fromstepstr}}\\
{}=  \mcons\, (\state_{\nat}\,(\lambda a . \pair{a}{  
\mcons\, (\state_{\nat} \,(\lambda a . \pair{a}{tts})\, (\lambda s . \pair{s + n}{s + 2n + 1}))
})\, (\lambda s. \pair{s}{s + n}))
\\ \mbox{by definition of } \head \\
{}= \mcons\, (\state_{\nat}\,(\lambda a . \pair{a}{  
\mcons\, (\lambda s . \pair{\pair{s + n}{tts}}{s + 2n + 1})
})\, (\lambda s. \pair{s}{s + n})) \\
\mbox{by functor definition of } \state_{\nat}  \\
{}= \mcons\, (\lambda s. \pair{\pair{s}{\mcons\, (\lambda s . \pair{\pair{s + n}{tts}}{s + 2n + 1})}}{s + n}) \\ 
\mbox{by functor definition of } \state_{\nat} \\
\end{array} 
$$
It is clear by inspection that 
$$
\begin{array}{l}
\mcons\, (\lambda s. \pair{\pair{s}{(\mcons\, (\lambda s. \pair{\pair{s}{\fromstepstr\, (n + 2)}}{s + n + 1}))}}{s + n}) \neq \\
\qquad \mcons\, (\lambda s. \pair{\pair{s}{\mcons\, (\lambda s . \pair{\pair{s + n}{tts}}{s + 2n + 1})}}{s + n})
\end{array} 
$$
without expanding $tts$ and $\fromstepstr\, (n + 2)$ any further.
\end{proof}

The issue is that the first monadic action (that of $\fromstepstr\, n$) is repeated twice in the monster resulting from the $\join$ operation. 
This indicates that monsters may still be monads if the underlying functor is an idempotent monad instead. 
We have yet to prove this, and we are also looking for a weaker condition on the underlying functor, as idempotency is a strong requirement.
