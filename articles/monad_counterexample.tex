\section{Monsters are Not Monads}


\vcomm{Counterexample showing that taking the diagonal as join doesn't work}

We show here that the naive and seemingly most natural way of defining monad operations on monsters doesn't satisfy the monad laws. \\

The Monad class has two methods: $\return$ and $\bind$:
$$
\begin{array}{l}
\return : A \rightarrow M\, A \\
(\bind) : M\, A \rightarrow (A \rightarrow M\, B) \rightarrow M\, B
\end{array}
$$
If a functor is a monad, it is an applicative functor. The correct definition for the $\return$ operation is always the same as that of $\apure$, and visa-versa. \\

A more categorical way of defining a monad is in terms $\join : M\,(M\, A) \rightarrow M\, A$. This operation 'flattens' two layers of a functor into one, and can be thought of as similar to a monoid operation.

$$
\join : M\,(M\, A) \rightarrow M\, A
$$

The $\bind$ operation can be defined in terms of $\join$, and visa-versa.
$$
\begin{array}{l}
(\bind\, f) = \join \comp M f \\
\join = (\bind\,\id)
\end{array}
$$

By showing that a natural definition of $\join$ doesn't satisfy the monad laws, we also show that the corresponding definition of $\bind$ doesn't either. 

$$
\begin{array}{l}
\fromstr : \nat \rightarrow \stream{\Id}\,\nat \\
\fromstr n = \mcons (\return \pair{n}{\fromstr (n + 1)})
\end{array}
$$

$$
\begin{array}{l}
\fromstepstr : \nat \rightarrow \stream{\state}\,\nat \\
\fromstepstr n = \mcons (\lambda s. \pair{\pair{s}{\fromstepstr (n + 1)}}{s + n})
\end{array}
$$