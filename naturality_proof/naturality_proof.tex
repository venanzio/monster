\documentclass{article}

\usepackage{amssymb,amsmath,amsthm}
\usepackage{pstricks,pst-node}
\usepackage{listings}
\usepackage{hyperref}

\usepackage{tikz}
\usetikzlibrary{trees}
\title{Monsters can be Monads in Only One Way}

\usepackage{notations_2}

\newtheorem{definition}{Definition}
\newtheorem{theorem}[definition]{Theorem}
\newtheorem{proposition}[definition]{Proposition}
\newtheorem{lemma}[definition]{Lemma}
\newtheorem{example}[definition]{Example}
\newtheorem{conjecture}[definition]{Conjecture}

\begin{document}
\bibliographystyle{plain}

\newcommand{\tabulatef}{\mathsf{tabulate}}
\newcommand{\indexf}{\mathsf{index}}
\newcommand{\sfork}{\mathsf{sfork}}
\newcommand{\Set}{\mathsf{Set}}

\maketitle

\textit{Sidenote}: In the following, $M_i, N_i, E_i$ are functors equipped with monad definitions, and $A,B,X,Y,R,S$ are objects (types/sets)\\

\textit{Sidenote 2}: Using Yoneda, you can derive a nice isomorphism if $M$ is representable:
$$
\begin{array}{ll}
	\stream{M}^2 \Rightarrow \stream{M} \\
	\cong \text{[$\stream{M}$ representable if $M$ is]} \\
	([R_M] \to ([R_M] \to -)) \Rightarrow ([R_M] \to -) \\
	\cong \text{[Yoneda]} \\
	([R_M] \to [R_M] \times [R_M])
\end{array}
$$
It seems from this that we could have used any representable functor $M$ to prove the following, and not just $\Id$.

\vspace{1cm}

To prove this statement (the title of the document), we need a single pivotal lemma.

\begin{lemma}\label{lemma:stream_monad}
The pure stream functor $\stream{\Id}: \Set \to \Set$, defined as the final coalgebra of the functor $F\,X = A \times X$ can be made a monad in only one way.
\end{lemma}
\begin{proof}
We start with the well-known fact that $\stream{\Id}\,A \cong \nat \rightarrow A$ (think of $n :\nat$ as indexing the elements in the stream). From this we can characterise $\eta^{\stream{\Id}}$ and $\mu^{\stream{\Id}}$ as:
$$
\begin{array}{ll} \ \\
\eta^{\stream{\Id}} : \Id \Rightarrow (\nat \to -) \\
\mu^{\stream{\Id}} : (\nat \to (\nat \to -)) \Rightarrow (\nat \to -)
\end{array}
$$
Where $(\nat \to -)$ is the covariant reader functor at $\nat$. Using the Yoneda lemma, we can simplify these types. Starting with $\eta^{\stream{\Id}}$:
$$
\begin{array}{ll} \ \\
\Id \Rightarrow (\nat \to -) \\
\cong \text{[$\Id$ is representable by the singleton set $1$]} \\
(1 \to -) \Rightarrow (\nat \to -) \\
\cong \text{[Yoneda embedding]} \\
\nat \to 1
\end{array}
$$
By terminality, all functions $A \to 1$ in $\Set$ are unique. Since $\nat \to 1$ only has one element, so does $\Id \Rightarrow (\nat \to -)$ namely $\eta^{\stream{\Id}}$. Practically, this means that all ways you can define $\eta^\Id$ are equal. We choose the definition:
$$
(\eta^{\stream{\Id}})_A\, a = \lambda \_ . a
$$

Following the proof of the Yoneda lemma in the case of $(\nat \to -) \to (\nat \times \nat \to -)) \cong \nat \times \nat \to \nat$ you end up with:
$$
\begin{array}{ll}
    \phi : (\nat \to -) \to (\nat \to (\nat \to -)) \\
    u : \nat \times \nat \to \nat \\
    f : \nat \to \nat \\
    f \comp u = \phi_\nat\,f
\end{array}
$$
This gives us the two equations:
 $$
\begin{array}{ll}
    f \comp \pi_1 = (\stream{\Id}\eta^{\stream{\Id}})_A\, f \\
    f \comp \pi_2 = (\eta^{\stream{\Id}}\stream{\Id})_A\, f \\
\end{array}
$$
Which, since we know the definition of $\eta^{\stream{\Id}}$ and the way that $\stream{\Id}$ maps functions, allows us to calculate $\pi_1$, $\pi_2$:
$$
\begin{array}{ll}
    f \comp \pi_1 = (\stream{\Id}\eta^{\stream{\Id}})_A\, f \\
    = \text{[definition of $\stream{\Id}\eta^{\stream{\Id}}$]} \\
    \stream{\Id}((\eta^{\stream{\Id}})_A)\, f \\
    = \text{[definition of $\stream{\Id}$ function mapping]} \\
    (\eta^{\stream{\Id}})_A \comp f  \\
    = \text{[naturality of $\eta^{\stream{\Id}}$]} \\
    \stream{\Id}\,f \comp (\eta^{\stream{\Id}})_\nat \\
    = \text{[definition of of $\eta^{\stream{\Id}}$]} \\
    \stream{\Id}\,f \comp (\lambda\,a.\lambda\,b.a) \\
    = \text{[definition of $\stream{\Id}$ function mapping]} \\
    (f\,\comp) \comp (\lambda\,a.\lambda\,b.a) \\
    = \text{[definition of $\comp$]} \\
    \lambda\,x.(f\,\comp)\,((\lambda\,a.\lambda\,b.a)\,x) \\
    = \\
    \lambda\,x.f \comp (\lambda\,b.x) \\
    = \text{[definition of $\comp$]} \\
    \lambda\,x.\lambda\,y. f\,((\lambda\,b.x)\, y) \\
    = \\
    \lambda\,x.\lambda\,y. f\,x \\
    = \text{[currying]} \\
    \lambda\pair{x}{y}. f\,x \\\\
    f\,(\pi_1\,\pair{n}{m}) = (\lambda\pair{x}{y}. f\,x) \pair{n}{m} = f\,n
\end{array}
$$
$$
\begin{array}{ll}
    f \comp \pi_2 = (\eta^{\stream{\Id}}\stream{\Id})_A\, f \\
    = \text{[definition of $\eta^{\stream{\Id}}\stream{\Id}$]} \\
    (\eta^{\stream{\Id}})_{\stream{\Id}\,A}\, f \\
    = \text{[definition of $\eta^{\stream{\Id}}$]} \\
    \lambda \_ . f \\\\
    
    f\,(\pi_2\,\pair{n}{m}) = (\lambda\_ . f) \pair{n}{m} = f\,m
\end{array}
$$

Following again the Yoneda lemma with $\phi : (\nat \to -) \Rightarrow (\nat \to -)$ and $u : \nat \to \nat$, we get the equation:
$$
f \comp u = (\id_{\stream{\Id}})_A\, f = f \\
$$
Which, since $f$ is arbitrary, can only imply $u = \id_\nat$. \\

We have shown using the Yoneda lemma that $\stream{\Id}\eta^{\stream{\Id}}$, $\eta^{\stream{\Id}}\stream{\Id}$ and $\id_{\stream{\Id}}$ are \emph{uniquely defined} by $\pi_1$, $\pi_2$ and $\id_\nat$ respectively. \\

Following again the Yoneda lemma with $\phi : (\nat \times \nat \to -) \Rightarrow (\nat \to -)$ and $u : \nat \to \nat \times \nat$, we get this equation by choosing $\phi = \mu^{\stream{\Id}}$:
$$
f \comp u = (\mu^{\stream{\Id}})_A\, f \\
$$

We now look at the monad laws for $\mu^{\stream{\Id}}$:
$$
\begin{array}{ll}
\mu^{\stream{\Id}} \comp \eta^{\stream{\Id}}\stream{\Id} = \id_{\stream{\Id}} \\
\mu^{\stream{\Id}} \comp \stream{\Id}\eta^{\stream{\Id}} = \id_{\stream{\Id}}
\end{array}
$$
By looking at the components of these, we can derive two equations:
$$
\begin{array}{ll}
(\id_{\stream{\Id}})_A\,f = ((\mu^{\stream{\Id}})_A \comp (\eta^{\stream{\Id}}\stream{\Id})_A)\,f  \\
= \text{[definition of $\comp$ and $\pi_2$]} \\
(\mu^{\stream{\Id}})_A (f \comp \pi_2) \\
= \text{[definition of $u$ and $\id_\nat$]} \\
(f \comp \pi_2) \comp u = f \comp \id_\nat = f \\\\

\pi_2 \comp u = \id_\nat
\end{array}
$$
$$
\begin{array}{ll}
(\id_{\stream{\Id}})_A\,f = ((\mu^{\stream{\Id}})_A \comp (\stream{\Id}\eta^{\stream{\Id}})_A)\,f \\
= \text{[definition of $\comp$ and $\pi_1$]} \\
(\mu^{\stream{\Id}})_A (f \comp \pi_1) \\
= \text{[definition of $u$ and $\id_\nat$]} \\
(f \comp \pi_1) \comp u = f \comp \id_\nat = f \\\\

\pi_1 \comp u = \id_\nat
\end{array}
$$

This gives us:
$$
\begin{array}{ll}
\pi_1 (u\, n) = \id_\nat\,n \\
\pi_2 (u\, n) = \id_\nat\,n 
\end{array}
$$
Which explicitly defines $u = \lambda\,n.\pair{n}{n}$. From this we can calculate $\mu^{\stream{\Id}}$:
$$
\begin{array}{ll}
(\mu^{\stream{\Id}})_A\, f = f \comp u \\
= \\
f \comp \lambda\,x.\pair{x}{x} \\
= \\
\lambda n.f \pair{n}{n}
\end{array}
$$
Which clearly defines $\mu^{\stream{\Id}}$ as taking the diagonal of a stream of streams.
\end{proof}

Given this lemma, we now conjecture that from the functoriality of $\stream{M}$ in its argument $M$, $\mu^{\stream{M}}$ can be defined in only one way - this way can be calculated from the following naturality condition:

\begin{lemma}\label{lemma:naturality_of_mstr}
$$\stream{}(\eta^M) \bullet \mu^{\stream{\Id}} = \mu^{\stream{M}} \bullet (\stream{}(\eta^M) \comp \stream{}(\eta^M))$$
\end{lemma}
\begin{proof}
This follows directly from chasing the below commuting square, derived from the assertion that $\stream{M}$ is functorial in $M$.
$$
\setlength\arraycolsep{30pt}
\begin{array}{cc} \ \\
\Rnode{sid}{\stream{\Id}} & \Rnode{sid2}{(\stream{\Id})^2} \\[30pt]
\Rnode{sm}{\stream{M}} & \Rnode{sm2}{(\stream{M})^2}  \\ \ 
\end{array}
\psset{nodesep=5pt,arrows=->}
\ncline{sid2}{sid} \taput{\mu^{\stream{\Id}}}
\ncline{sm2}{sm} \tbput{\mu^{\stream{M}}}
\ncline{sid}{sm}  \tlput{\stream{}(\eta^M)}
\ncline{sid2}{sm2}  \trput{\stream{}(\eta^M) \comp \stream{}(\eta^M)}
$$
\end{proof}

We define $\stream{}(\phi)$, where $\phi: M_1 \Rightarrow M_2$ as:
\begin{align*}
	\stream{}(\phi)_A =\,&\mcons \circ \phi_{A \times \stream{M_2}} \circ\, \\
	&M_1(\lambda (h, t).\pair{h}{\stream{}(\phi)_A\, t})\, \circ\, \\ 
	&\uncons
\end{align*}
The above is the 'natural' way of defining the lifting of a natural transformation by the functor $\stream{}$??? (how to prove this?) \\

Given Lemma \ref{lemma:stream_monad} we know that the definition of $\mu^{\stream{\Id}}$ is unique up to isomorphism. Thus our chosen definition of $\mu^{\stream{\Id}}$ is the following:
\begin{align*}
	\mu^{\stream{\Id}}_A =\, &\mcons \circ \mu^\Id_{A \times \stream{\Id}A}\, \circ\, \\
	&\Id\,(\lambda (hs, ts). (\Id\,(\lambda \bar{hs} . \pair{\bar{hs}}{(\mu^{\stream{\Id}}_A \comp \stream{\Id}\,\tail)\,ts}) \circ \head)\,hs)\, \circ\, \\
	&\uncons
\end{align*}

The long function in the middle is just a fancy way of applying functions to the heads and tails of a stream of streams (*an uncons'd one) underneath its various functor layers.

Lets define a rather general helper function to assist with this:
$$
\begin{array}{ll}
\sfork : ((M_1\, A \rightarrow N_1\, X) \times  (M_2\, B \rightarrow N_2\, Y)) \to 
\\ \qquad\qquad M_1\, A \times M_2\, B \to N_1\,(X \times N_2\,Y) \\
\sfork\,\pair{f}{g} = \lambda (h, t). (N_1\,(\lambda \bar{h} . \pair{\bar{h}}{g\,t}) \circ f)\,h
\end{array}
$$

Along with an odd lemma about composing them:
\begin{lemma}\label{lemma:sfork_comp}
For functions: 
$$
\begin{array}{ll}
f : M_1\,A \to \Id\,X & \qquad
k : \Id\,X \to E_1\,R  \\
g : M_2\,B \to N_2\,Y &\qquad
l : N_2\,Y \to E_2\,S
\end{array}
$$
The following holds:
$$
\sfork\,\pair{k}{l} \comp \sfork\,\pair{f}{g} =  \sfork\,\pair{k \comp f}{l \comp g}
$$
\end{lemma}
\begin{proof}
This is clear by unfolding the definition of $\sfork$ and considering that $\Id\,X = X$ (I can fill this proof in later but it should be really straight-forward)
\end{proof}

Which allows us to rewrite the above definitions. Since the $\Id$ functors mapping, on both objects and morphisms, are just reflections, we will be slightly loose with adding/removing the $\Id$ functor 'label' as required in the proof (as in, not spelling out each step in which we use this isomorphism due to its triviality).
\begin{align*}
	\stream{}(\phi)_A =\,&\mcons \circ \phi_{A \times \stream{M_2}} \circ\, \\
	&\Id\,(\sfork\,\pair{\id}{\stream{}(\phi)_A})\, \circ\, \\ 
	&\uncons
\end{align*}
\begin{align*}
	\mu^{\stream{\Id}}_A =\, &\mcons \circ \mu^\Id_{A \times \stream{\Id}A}\, \circ\, \\
	&\Id\,(\sfork\,\pair{\head}{\mu^{\stream{\Id}}_A \comp \stream{\Id}\,\tail})\, \circ\, \\
	&\uncons
\end{align*}

This gives us a more detailed breakdown of Lemma \ref{lemma:naturality_of_mstr}:
$$
\begin{array}{ll}
	\mu^{\stream{M}} \bullet (\stream{}(\eta^M) \comp \stream{}(\eta^M)) \\
	= \\
	\mcons \circ \eta^M_{A \times \stream{M}} \circ\, \Id\,(\sfork\,\pair{\id}{\stream{}(\eta^M)_A})\, \circ\, \\ 
	 \uncons \circ \mcons\, \circ\, \\
	 \mu^\Id_{A \times \stream{\Id}A} \circ \Id\,(\sfork\,\pair{\head}{\mu^{\stream{\Id}}_A \comp \stream{\Id}\,\tail})\, \circ\, \\
	 \uncons \\
	 = \text{[Lambek's lemma, $\uncons \circ \mcons = \id$]}\\
	\mcons \circ \eta^M_{A \times \stream{M}} \circ\, \Id\,(\sfork\,\pair{\id}{\stream{}(\eta^M)_A})\, \circ\, \\ 
	 \mu^\Id_{A \times \stream{\Id}A} \circ \Id\,(\sfork\,\pair{\head}{\mu^{\stream{\Id}}_A \comp \stream{\Id}\,\tail})\, \circ\, \\
	 \uncons \\
	 = \text{[naturality of $\mu^M$]}\\
	\mcons \circ \eta^M_{A \times \stream{M}} \circ\, \mu^\Id_{A \times \stream{M}A}\, \circ\, \\ 
	 (\sfork\,\pair{\id}{\stream{}(\eta^M)_A}) \circ \Id\,(\sfork\,\pair{\head}{\mu^{\stream{\Id}}_A \comp \stream{\Id}\,\tail})\, \circ\, \\
	 \uncons \\
	 = \text{[$\Id\,f = f$]}\\
	\mcons \circ \eta^M_{A \times \stream{M}} \circ\, \mu^\Id_{A \times \stream{M}A}\, \circ\, \\ 
	 \Id\,(\sfork\,\pair{\id}{\stream{}(\eta^M)_A}) \circ \Id\,(\sfork\,\pair{\head}{\mu^{\stream{\Id}}_A \comp \stream{\Id}\,\tail})\, \circ\, \\
	 \uncons \\
	 = \text{[Functoriality of $\Id$]}\\
	 \mcons \circ \eta^M_{A \times \stream{M}} \circ\, \mu^\Id_{A \times \stream{M}A}\, \circ\, \\ 
	 \Id\,(\sfork\,\pair{\id}{\stream{}(\eta^M)_A} \circ \sfork\,\pair{\head}{\mu^{\stream{\Id}}_A \comp \stream{\Id}\,\tail})\, \circ\, \\
	 \uncons \\
	 = \text{[by Lemma \ref{lemma:sfork_comp}]}\\
	 \mcons \circ \eta^M_{A \times \stream{M}} \circ\, \mu^\Id_{A \times \stream{M}A}\, \circ\, \\ 
	 \Id\,(\sfork\,\pair{\head}{\stream{}(\eta^M)_A \comp \mu^{\stream{\Id}}_A \comp \stream{\Id}\,\tail})\, \circ\, \\
	 \uncons \\
	 = \text{[by Lemma \ref{lemma:naturality_of_mstr}]}\\
	 \mcons \circ \eta^M_{A \times \stream{M}} \circ\, \mu^\Id_{A \times \stream{M}A}\, \circ\, \\ 
	 \Id\,(\sfork\,\pair{\head}{(\mu^{\stream{M}} \bullet (\stream{}(\eta^M) \comp \stream{}(\eta^M)))_A \comp \stream{\Id}\,\tail})\, \circ\, \\
	 \uncons \\
\end{array}
$$

It seems that to proceed, we lift $(\stream{}(\eta^M) \comp \stream{}(\eta^M))$ up through the lambda functions until it is on the outside, at which point we should hopefully be able to cancel on both sides of the equation. 

This looks even more promising considering that the 'target' $\mu^{\stream{M}}$ is:
\begin{align*}
	\mu^{\stream{M}}_A =\, &\mcons \circ \mu^M_{A \times \stream{M}A}\, \circ\, \\
	&M\,(\sfork\,\pair{\head}{\mu^{\stream{M}}_A \comp \stream{M}\,\tail})\, \circ\, \\
	&\uncons
\end{align*}

Recapping:
$$
\begin{array}{ll}
	\mu^{\stream{M}} \bullet (\stream{}(\eta^M) \comp \stream{}(\eta^M)) \\
	= \\
	 \mcons \circ \eta^M_{A \times \stream{M}} \circ\, \mu^\Id_{A \times \stream{M}A}\, \circ\, \\ 
	 \Id\,(\sfork\,\pair{\head}{(\mu^{\stream{M}} \bullet (\stream{}(\eta^M) \comp \stream{}(\eta^M)))_A \comp \stream{\Id}\,\tail})\, \circ\, \\
	 \uncons \\
	 = \text{[definition of vertical composition of natural transformations]}\\
	 \mcons \circ \eta^M_{A \times \stream{M}} \circ\, \mu^\Id_{A \times \stream{M}A}\, \circ\, \\ 
	 \Id\,(\sfork\,\pair{\head}{(\mu^{\stream{M}}_A \comp (\stream{}(\eta^M) \comp \stream{}(\eta^M))_A \comp \stream{\Id}\,\tail})\, \circ\, \\
	 \uncons \\
	 = \text{[by magic]}\\
	 \mcons \circ \eta^M_{A \times \stream{M}} \circ\, \mu^\Id_{A \times \stream{M}A}\, \circ\, \\ 
	 \Id\,(\sfork\,\pair{\head}{(\mu^{\stream{M}}_A \comp \stream{M}\,\tail \comp (\stream{}(\eta^M) \comp \stream{}(\eta^M))_A})\, \circ\, \\
	 \uncons \\
\end{array}
$$

$$
\begin{array}{ll}
	\head : \stream{M} \Rightarrow M \\
	\head_A = M(\pi_1) \comp \uncons \\\\
	\tail : \stream{M} \Rightarrow \stream{M} \\
	\tail_A = \mcons \comp \mu^M_{A \times \stream{M}A} \comp M(\uncons \comp \pi_2) \comp \uncons 
\end{array}
$$
$$
\setlength\arraycolsep{30pt}
\begin{array}{cc} \ \\
\Rnode{sida}{\stream{\Id}} & \Rnode{sidb}{\stream{\Id}} \\[30pt]
\Rnode{sma}{\stream{M}} & \Rnode{smb}{\stream{M}}  \\ \ 
\end{array}
\psset{nodesep=5pt,arrows=->}
\ncline{sida}{sidb} \taput{\tail}
\ncline{sida}{sma} \tlput{\stream{}(\eta^M)}
\ncline{sma}{smb}  \tbput{\tail}
\ncline{sidb}{smb}  \trput{\stream{}(\eta^M)}
$$

%\begin{lemma}\label{lemma:}

\end{document}
