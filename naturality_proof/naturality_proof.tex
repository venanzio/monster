\documentclass{article}

\usepackage{amssymb,amsmath,amsthm}
\usepackage{pstricks,pst-node}
\usepackage{listings}
\usepackage{hyperref}

\usepackage{tikz}
\usetikzlibrary{trees}
\title{Monsters can be Monads in Only One Way}

\usepackage{notations_2}

\newtheorem{definition}{Definition}
\newtheorem{theorem}[definition]{Theorem}
\newtheorem{proposition}[definition]{Proposition}
\newtheorem{lemma}[definition]{Lemma}
\newtheorem{example}[definition]{Example}
\newtheorem{conjecture}[definition]{Conjecture}

\begin{document}
\bibliographystyle{plain}

\newcommand{\tabulatef}{\mathsf{tabulate}}
\newcommand{\indexf}{\mathsf{index}}
\newcommand{\sfork}{\mathsf{sfork}}
\newcommand{\Set}{\mathsf{Set}}

\maketitle

\textit{Sidenote}: In the following, $M_i, N_i, E_i$ are functors equipped with monad definitions, and $A,B,X,Y,R,S$ are objects (types/sets)

\vspace{1cm}

To prove this statement, we need a single pivotal lemma.

\begin{lemma}\label{lemma:stream_monad}
The pure stream functor $\stream{\Id}: \Set \to \Set$, defined as the final coalgebra of the functor $F\,X = A \times X$ can be made a monad in only one way.
\end{lemma}
\begin{proof}
$$\text{Insert funky Yoneda-based reasoning here}$$
\end{proof}

Given this lemma, we now conjecture that from the functoriality of $\stream{M}$ in its argument $M$, $\mu^{\stream{M}}$ can be defined in only one way - this way can be calculated from the following naturality condition:

\begin{lemma}\label{lemma:naturality_of_mstr}
$$\stream{}(\eta^M) \bullet \mu^{\stream{\Id}} = \mu^{\stream{M}} \bullet (\stream{}(\eta^M) \comp \stream{}(\eta^M))$$
\end{lemma}
\begin{proof}
This follows directly from chasing the below commuting square, derived from the assertion that $\stream{M}$ is functorial in $M$.
$$
\setlength\arraycolsep{30pt}
\begin{array}{cc} \ \\
\Rnode{sid}{\stream{\Id}} & \Rnode{sid2}{(\stream{\Id})^2} \\[30pt]
\Rnode{sm}{\stream{M}} & \Rnode{sm2}{(\stream{M})^2}  \\ \ 
\end{array}
\psset{nodesep=5pt,arrows=->}
\ncline{sid2}{sid} \taput{\mu^{\stream{\Id}}}
\ncline{sm2}{sm} \tbput{\mu^{\stream{M}}}
\ncline{sid}{sm}  \tlput{\stream{}(\eta^M)}
\ncline{sid2}{sm2}  \trput{\stream{}(\eta^M) \comp \stream{}(\eta^M)}
$$
\end{proof}

We define $\stream{}(\phi)$, where $\phi: M_1 \Rightarrow M_2$ as:
\begin{align*}
	\stream{}(\phi)_A =\,&\mcons \circ \phi_{A \times \stream{M_2}} \circ\, \\
	&(\lambda (h, t).\pair{h}{\stream{}(\phi)_A\, t})\, \circ\, \\ 
	&\uncons
\end{align*}
The above is the 'natural' way of defining the lifting of a natural transformation by the functor $\stream{}$??? (how to prove this?) \\

Given Lemma \ref{lemma:stream_monad} we know that the definition of $\mu^{\stream{\Id}}$ is unique up to isomorphism. Thus our chosen definition of $\mu^{\stream{\Id}}$ is the following:
\begin{align*}
	\mu^{\stream{\Id}}_A =\, &\mcons \circ \mu^\Id_{A \times \stream{\Id}A}\, \circ\, \\
	&\Id\,(\lambda (hs, ts). (\Id\,(\lambda \bar{hs} . \pair{\bar{hs}}{(\mu^{\stream{\Id}}_A \comp \stream{\Id}\,\tail)\,ts}) \circ \head)\,hs)\, \circ\, \\
	&\uncons
\end{align*}

The long function in the middle is just a fancy way of applying functions to the heads and tails of a stream of streams (*an uncons'd one) underneath its various functor layers.

Lets define a rather general helper function to assist with this:
$$
\begin{array}{ll}
\sfork : ((M_1\, A \rightarrow N_1\, X) \times  (M_2\, B \rightarrow N_2\, Y)) \to 
\\ \qquad\qquad M_1\, A \times M_2\, B \to N_1\,(X \times N_2\,Y) \\
\sfork\,\pair{f}{g} = \lambda (h, t). (N_1\,(\lambda \bar{h} . \pair{\bar{h}}{g\,t}) \circ f)\,h
\end{array}
$$

Along with an odd lemma about composing them:
\begin{lemma}\label{lemma:sfork_comp}
For functions: 
$$
\begin{array}{ll}
f : M_1\,A \to \Id\,X & \qquad
k : \Id\,X \to E_1\,R  \\
g : M_2\,B \to N_2\,Y &\qquad
l : N_2\,Y \to E_2\,S
\end{array}
$$
The following holds:
$$
\sfork\,\pair{k}{l} \comp \sfork\,\pair{f}{g} =  \sfork\,\pair{k \comp f}{l \comp g}
$$
\end{lemma}
\begin{proof}
This is clear by unfolding the definition of $\sfork$ and considering that $\Id\,X \cong X$ (I can fill this in later but it should be really straight-forward)
\end{proof}

Which allows us to rewrite the above definitions. Since the $\Id$ functors mapping, on both objects and morphisms, are isomorphisms, we will be slightly loose with adding/removing the $\Id$ functor 'label' as required in the proof (as in, not spelling out each step in which we use this isomorphism due to its triviality).
\begin{align*}
	\stream{}(\phi)_A =\,&\mcons \circ \phi_{A \times \stream{M_2}} \circ\, \\
	&\Id\,(\sfork\,\pair{\id}{\stream{}(\phi)_A})\, \circ\, \\ 
	&\uncons
\end{align*}
\begin{align*}
	\mu^{\stream{\Id}}_A =\, &\mcons \circ \mu^\Id_{A \times \stream{\Id}A}\, \circ\, \\
	&\Id\,(\sfork\,\pair{\head}{\mu^{\stream{\Id}}_A \comp \stream{\Id}\,\tail})\, \circ\, \\
	&\uncons
\end{align*}

This gives us a more detailed breakdown of Lemma \ref{lemma:naturality_of_mstr}:
$$
\begin{array}{ll}
	\mu^{\stream{M}} \bullet (\stream{}(\eta^M) \comp \stream{}(\eta^M)) \\
	= \\
	\mcons \circ \eta^M_{A \times \stream{M}} \circ\, \Id\,(\sfork\,\pair{\id}{\stream{}(\eta^M)_A})\, \circ\, \\ 
	 \uncons \circ \mcons\, \circ\, \\
	 \mu^\Id_{A \times \stream{\Id}A} \circ \Id\,(\sfork\,\pair{\head}{\mu^{\stream{\Id}}_A \comp \stream{\Id}\,\tail})\, \circ\, \\
	 \uncons \\
	 = \text{[Lambek's lemma, $\uncons \circ \mcons = \id$]}\\
	\mcons \circ \eta^M_{A \times \stream{M}} \circ\, \Id\,(\sfork\,\pair{\id}{\stream{}(\eta^M)_A})\, \circ\, \\ 
	 \mu^\Id_{A \times \stream{\Id}A} \circ \Id\,(\sfork\,\pair{\head}{\mu^{\stream{\Id}}_A \comp \stream{\Id}\,\tail})\, \circ\, \\
	 \uncons \\
	 = \text{[naturality of $\mu^M$]}\\
	\mcons \circ \eta^M_{A \times \stream{M}} \circ\, \mu^\Id_{A \times \stream{M}A}\, \circ\, \\ 
	 (\sfork\,\pair{\id}{\stream{}(\eta^M)_A}) \circ \Id\,(\sfork\,\pair{\head}{\mu^{\stream{\Id}}_A \comp \stream{\Id}\,\tail})\, \circ\, \\
	 \uncons \\
	 = \text{[$\Id\,f \cong f$]}\\
	\mcons \circ \eta^M_{A \times \stream{M}} \circ\, \mu^\Id_{A \times \stream{M}A}\, \circ\, \\ 
	 \Id\,(\sfork\,\pair{\id}{\stream{}(\eta^M)_A}) \circ \Id\,(\sfork\,\pair{\head}{\mu^{\stream{\Id}}_A \comp \stream{\Id}\,\tail})\, \circ\, \\
	 \uncons \\
	 = \text{[Functoriality of $\Id$]}\\
	 \mcons \circ \eta^M_{A \times \stream{M}} \circ\, \mu^\Id_{A \times \stream{M}A}\, \circ\, \\ 
	 \Id\,(\sfork\,\pair{\id}{\stream{}(\eta^M)_A} \circ \sfork\,\pair{\head}{\mu^{\stream{\Id}}_A \comp \stream{\Id}\,\tail})\, \circ\, \\
	 \uncons \\
	 = \text{[by Lemma \ref{lemma:sfork_comp}]}\\
	 \mcons \circ \eta^M_{A \times \stream{M}} \circ\, \mu^\Id_{A \times \stream{M}A}\, \circ\, \\ 
	 \Id\,(\sfork\,\pair{\head}{\stream{}(\eta^M)_A \comp \mu^{\stream{\Id}}_A \comp \stream{\Id}\,\tail})\, \circ\, \\
	 \uncons \\
	 = \text{[by Lemma \ref{lemma:naturality_of_mstr}]}\\
	 \mcons \circ \eta^M_{A \times \stream{M}} \circ\, \mu^\Id_{A \times \stream{M}A}\, \circ\, \\ 
	 \Id\,(\sfork\,\pair{\head}{(\mu^{\stream{M}} \bullet (\stream{}(\eta^M) \comp \stream{}(\eta^M)))_A \comp \stream{\Id}\,\tail})\, \circ\, \\
	 \uncons \\
\end{array}
$$

It seems that to proceed, we lift $(\stream{}(\eta^M) \comp \stream{}(\eta^M))$ up through the lambda functions until it is on the outside, at which point we should hopefully be able to cancel on both sides of the equation. 

This looks even more promising considering that the 'target' $\mu^{\stream{M}}$ is:
\begin{align*}
	\mu^{\stream{M}}_A =\, &\mcons \circ \mu^M_{A \times \stream{M}A}\, \circ\, \\
	&M\,(\sfork\,\pair{\head}{\mu^{\stream{M}}_A \comp \stream{M}\,\tail})\, \circ\, \\
	&\uncons
\end{align*}


\end{document}
