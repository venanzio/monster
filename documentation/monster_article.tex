\documentclass{article}

\usepackage{amssymb,amsmath,amsthm}
\usepackage{listings}
\usepackage{pstricks,pst-node}

\usepackage[utf8]{inputenc}
\usepackage[english]{babel}

\title{Programming with Monsters}
\author{Venanzio Capretta and Christopher Purdy}


\usepackage{notations}

\newtheorem{definition}{Definition}
\newtheorem{theorem}{Theorem}
\newtheorem{proposition}{Proposition}
\newtheorem{example}{Example}




%format 

\begin{document}
\bibliographystyle{plain}

\maketitle

\begin{abstract}
A monadic stream is a potentially infinite sequence of values in which a monadic action is triggered at each step before the generation of the next element.
A monadic action consists of some functional {\em container} inside which the unfolding of the stream takes place.
Examples of monadic actions are: executing some input/output interactions, cloning the process into several parallel computations, executing transformations on an underlying state, terminating the sequence.

We develop a library of definitions and universal combinators to program with monadic streams and we prove several mathematical results about their behaviour.

We define the type of Monadic Streams (MonStr), dependent on two arguments: the underlying monad and the type of elements of the sequence.
We use the following terminology: a monadic stream with underlying monad $m$ is called an $m$-monster.
The definition itself doesn't depend on the fact that the underlying operator is a monad.
We define it generally: some of the operators can be defined without any assumptions, some others only need the operator to be a Functor or Applicative Functor.
A different set of important combinators and theoretical results follow from the assumption that the operator is a Co-Monad rather than a Monad.

We instantiate the abstract MonStr type with several common monads (Maybe, List, State, IO) and show that we obtain well-known data structures.
Maybe-monsters are lazy lists, List-monsters are non-well-founded finitely branching trees, IO-monsters are interactive processes.
We prove equivalences between the traditional data types and their MonStr versions: the MonStr combinators instantiate to traditional operations.

Under some assumptions on the underlying functor/monad, we can prove that the MonStr type is also a Functor, and Applicative, or a Monad, giving us access to the special methods and notations of those type classes.
\end{abstract}

\section{Introduction}

Intuitive explanation of Monadic Streams and motivation.

\section{Monadic Streams}

A monadic stream is a sequence of values in which every stage is obtained by triggering a monadic action.
If $\sigma$ is such a stream, it will consist of an action for a certain monad $M$ that, when executed, will return a head (first element) and a tail (continuation of the stream).
This process can be continued in a non-well-founded way: streams constitute a coinductive type.

Formally the type of streams over a monad $M$ (let's call them $M$-monsters) is defined as:
$$
\begin{array}[t]{l}
\codata\;
\stream{M,A}:\set\\
\quad \mcons_M: M\,(A\times \stream{M,A})\rightarrow\stream{M,A}.
\end{array}
$$
Categorically, we can see this type as the final coalgebra of the functor $F\,X = M\,(A\times X)$.

It is important to make two observations about $M$.

First, $M$ does not need to be a monad for the definition to make sense.
In fact we will obtain several interesting results when $M$ satisfies weaker conditions, for example being just a functor.
So we will take $M$ to be any type operator (but see second observation) and we will explicitely state what properties we assume about it.
The most important instances are monads and it is convenient to use the facilities of monadic notation in programming and monad theory in reasoning.

The second observation is that it is not guaranteed in general that the $\codata$ type is well-defined.
Haskell will accept the definition when $M$ is any functor, but mathematically the type is well defined only when the functor $F\,X = M\,(A\times X)$ has a final coalgebra.

\begin{definition}\label{def:fincoal}
For any functor $F$, a {\em coalgebra} for $F$ is pair $\langle A,\alpha\rangle$ consisting of a type $A$ and a function $\alpha:A\rightarrow FA$.

We say that $\langle A,\alpha\rangle$ is a {\em final $F$-coalgebra} if, 
for every coalgebra $\langle X,\xi:X\rightarrow FX\rangle$, 
there is a unique coalgebra morphism between the the two coalgebras:
$f:\langle X,\xi\rangle \to \langle A,\alpha\rangle$.
Such a morphism is a function between the types that commutes with the coalgebra functions:
$$
\setlength\arraycolsep{30pt}
\begin{array}{cc} \ \\
\Rnode{a}{A} & \Rnode{fa}{FA} \\[30pt]
\Rnode{x}{X} & \Rnode{fx}{FX} \\ \ 
\end{array}
\psset{nodesep=5pt,arrows=->}
\ncline{a}{fa} \taput{\alpha}
\ncline{x}{fx} \tbput{\xi}
\ncline[linestyle=dotted]{x}{a}  \tlput{f}
\ncline{fx}{fa} \trput{Ff}
\qquad \alpha \circ f = Ff \circ \xi.
$$
\end{definition}

This definition means that we can define function into a coinductive type by giving a coalgebra.
For example, consider the data type of pure streams of natural numbers:
$$
\begin{array}[t]{l}
\codata\;
\stream{A}:\set\\
\quad (\scons): \nat\times \stream{\nat} \rightarrow\stream{\nat}.
\end{array}
$$
The type $\stream{\nat}$ is the final coalgebra of the functor $F(X) = \nat\times X$.

We can define a function into $\stream{A}$ by defining a coalgebra on the domain type.
For example, if we want to define a function that maps any natural number $n$ to the stream of numbers starting from $n$, $n\scons (n+1) \scons (n+2) \scons (n+3) \scons \cdots$, we can do it by the following coalgebra on $\nat$:
$$
\begin{array}{l}
\xi : \nat \rightarrow \nat\times \nat\\
\xi\,n = \langle n,n+1\rangle
\end{array}
$$
(Note that the target type of the coalgebra is $F(\nat) = \nat\times\nat$: The first $\nat$ is the parameter of the functor, while the second $\nat$ is the carrier of the coalgebra.)

In practical programming, we often let the coalgebra $\xi$ be implicit by directly defining the function $f$ recursively.
For example, the function above would be defined as:
$$
\begin{array}{l}
f: \nat \rightarrow \stream{\nat}\\
f\,n = n \scons f\,(n+1)
\end{array}
$$
Here the presence of the parameter $n$ and the argument of the recursive call $n+1$ implicitely give the coalgebra $n \mapsto \langle n, n+1\rangle$.

\begin{vcomment}
Talk about definition by guarded recursion.
Show specific examples of definition by guarded recursion for monsters.
\end{vcomment}




The definition of $\stream{M,A}$ is not meaningful for all $M$s, because the final coalgebra may not exist or not be unique.
A useful result is that a functor has a final coalgebra if it is a container \cite{AAG:2005}, and $F$ is a container if $M$ is \cite{capretta/fowler:2017}.
This is the case for all the instances that we consider (but there are well known counterexamples, like the powerset functor and the continuation functor).

Therefore we will silently assume that $M$ is a container and that the final coalgebra exists.
Cofinality means that we can define functions into the coalgebra by {\em corecursion} and we can prove properties of its elements by {\em coinduction}.

\section{Examples}

Instantiations with Identity (Pure Streams), Maybe (Lazy Lists), List (Trees), IO (Processes), State (Stateful Streams).

Definition of the {\em standard} function on those data types in a general way.

\section{Instances of Functor, Applicative, Monad}

Show that monadic streams are an instance of these three classes, under some assumptions about the underlying monad.

So far we determined that we think that MonStr satisfies the monad laws if the underlying monad is commutative and idempotent \cite{idempotent_monads}. (Proof?)

What about Applicative?

Give conterexamples where it isn't a monad (using State).

Also examples when it is a monad even if the underlying monad is not commutative and idempotent: with List as underlying monad we obtain Trees, which are a monad. Can this be generalized.

\subsection{Functor instance in Haskell}

\begin{haskell}
unwrapMS :: MonStr m a -> m (a, MonStr m a)
unwrapMS (MCons m) = m

transformMS :: Functor m => (a -> MonStr m a -> (b, MonStr m b)) ->
                                  MonStr m a -> MonStr m b
transformMS f s = MCons $ fmap (\(h,t) -> f h t) (unwrapMS s)

instance Functor m => Functor (MonStr m) where
   -- fmap :: (a -> b) -> MonStr m a -> MonStr m b
   fmap f = transformMS (\a s -> (f a, fmap f s))
\end{haskell}

\subsubsection{Functor proof}

Proof that \hcode{fmap id == id}

\begin{haskell}
fmap id = transformMS (\a s -> id a, fmap id s)
-- definition of `transformMS`
= \s -> MCons $ fmap (\(h,t) -> (\a s -> (id a, fmap id s)) h t) (unwrapMS s)
-- application of `(\a s -> (id a, fmap id s))` to `h t`
= \s -> MCons $ fmap (\(h,t) -> (id h, fmap id t)) (unwrapMS s)
-- take as assumption that `fmap id t == id t` - coinductive hypothesis?
= \s -> MCons $ fmap (\(h,t) -> (id h, id t)) (unwrapMS s)
-- definition of `id`
= \s -> MCons $ fmap (\(h,t) -> (h,t)) (unwrapMS s)
-- definition of `id`
= \s -> MCons $ fmap id (unwrapMS s)
-- `unwrapMS` returns an element in a functor, so `fmap id == id` in this case
= \s -> MCons $ id (unwrapMS s)
-- application of `id`
= \s -> MCons $ (unwrapMS s)
-- MCons and unwrapMS are inverses
= \s -> id s 
= id
(*\hfill$\qed$*)
\end{haskell}

Proof that \verb+fmap\ (f \circ g) == (fmap\ f) \circ (fmap\ g)+

To simplify notation in the proof, we introduce the notation \verb+tr(f)+ for the expression \verb+\(h,t) -> (f h, fmap f t)+.

We're going to use the fact that the underlying operator \verb+m+ is a functor and so satisfies the functor laws, and also that the pairing operator \verb+(,)+ is functorial in both arguments.

By definition of \verb+fmap+ for monsters, we have that
\begin{lstlisting}
fmap f s = transformMS (\ a s -> (f a, fmap f s)) s
         = MCons $ fmap (\ (h,t) -> (\a s -> (f a, fmap f s) h t)) (unwrapMS s)
         = MCons $ fmap (\ (h,t) -> (f h, fmap f t)) (unwrapMS s)
         = MCons $ fmap tr(f) (unwrapMS s)
\end{lstlisting}

We also use the fact that the monster constructor \verb+MCons+ and the function \verb+unwrapMS+ that removes it are inverse of each other.

\begin{lstlisting}
fmap (f . g) = transformMS (\a s *'$\to$'* (f . g) a, fmap (f . g) s)
-- definition of `transformMS`
= \s *'$\to$'* MCons $ fmap (\(h,t) *'$\to$'* (\a s *'$\to$'* ((f . g) a, fmap (f . g) s)) h t) (unwrapMS s)
-- application of `(\a s *'$\to$'* ((f . g) a, fmap (f . g) s))` to `h t`
= \s *'$\to$'* MCons $ fmap (\(h,t) *'$\to$'* ((f . g) h, fmap (f . g) t)) (unwrapMS s)`
-- take as assumption that `fmap (f . g) t == (fmap f . fmap g) t` - coinductive hypothesis?
= \s *'$\to$'* MCons $ fmap (\(h,t) *'$\to$'* ((f . g) h, (fmap f . fmap g) t)) (unwrapMS s)
-- By fuctoriality of pairing in both components
= \s *'$\to$'* MCons $ fmap (tr(f) . tr(g)) (unwrapMS s)
-- By functoriality of the underlying operator m
= \s *'$\to$'* MCons $ (fmap tr(f)) . (fmap tr(g)) (unwrapMS s)
-- MCons and unwrapMS are inverses
= \s *'$\to$'* MCons $ fmap tr(f)) $ unwrapMS $ MCons $ fmap tr(g) (unwrapMS s)
= \s *'$\to$'* MCons $ fmap tr(f)) $ unwrapMS $ fmap g s
= \s *'$\to$'* fmap f $ fmap g s
= fmap f . fmap g
\end{lstlisting}

\subsection{Comonad instance in Haskell}

\begin{lstlisting}
instance Comonad w *'$\Rightarrow$'* Comonad (MonStr w) where
  --extract :: MonStr w a *'$\to$'* a
  extract = extract . headMS
  
  --duplicate :: MonStr w a *'$\to$'* MonStr w (MonStr w a)
  duplicate s = MCons $ fmap (\(h,t) *'$\to$'* (s, duplicate t)) (unwrapMS s)
\end{lstlisting}

\verb+headMS+ returns the first element of the first pair in the monster, wrapped in the underlying functor. Since this functor is required to be a comonad, extract can be used to return the wrapped values inside.

We introduce \verb+mm(s)+ for the "monster matrix" formed by a monadic stream \verb+s+ - this is defined as a monster where the first element is \verb+s+, the second is the tail of \verb+s+, the third is the tail of the tail of \verb+s+, and so on.

The \verb+duplicate+ instance for monadic streams forms this "monster matrix": \verb+duplicate s = mm(s)+

\subsubsection{Comonad proof}

Proof that \verb+extract . duplicate == id+

\begin{lstlisting}
extract . duplicate = \s *'$\to$'* (extract . headMS) . duplicate $ s
= \s *'$\to$'* (extract . headMS) (duplicate s)
-- duplicate forms a monster matrix
= \s *'$\to$'* (extract . headMS) mm(s)
-- the first element of the monster matrix is the original stream, as per the definition above.
-- taking the head returns the original stream guarded by the operator m, and extract then retrives 
-- s due to m being a comonad
= \s *'$\to$'* s
= id
\end{lstlisting}

Proof that \verb+fmap extract . duplicate == id+

\begin{lstlisting}
fmap extract . duplicate = \s *'$\to$'* fmap (extract . headMS) mm(s)
-- by functoriality of underlying operator m
= \s *'$\to$'* (fmap extract) . (fmap headMS) $ mm(s)
-- fmapping headMS replaces the monster at each step in the monster matrix with its
-- head, inside the operator. extract then removes these comonadic operators, resulting 
-- in the same monster that we started with
= \s *'$\to$'* s
= id
\end{lstlisting}

Proof that \verb+duplicate . duplicate == fmap duplicate . duplicate+

\section{Applications}

Show a few applications and motivate the usefulness of Monadic Streams.

\section{Conclusions}

Discussion of related literature.

Open problems.

Future Work.

\bibliography{biblio.bib}

\end{document}
