\section{Instances of Functor, Applicative, Monad}

Show that monadic streams are an instance of these three classes, under some assumptions about the underlying monad.

So far we determined that we think that MonStr satisfies the monad laws if the underlying monad is commutative and idempotent \cite{idempotent_monads}. (Proof?)

What about Applicative?

Give conterexamples where it isn't a monad (using State).

Also examples when it is a monad even if the underlying monad is not commutative and idempotent: with List as underlying monad we obtain Trees, which are a monad. Can this be generalized.

This section outlines the Functor, Applicative, and Comonad instances, proving that the functions defined on monadic streams to implement these satisfy the relevant laws. We will prove these with a mixture of equational reasoning (using Haskell), and categorical reasoning.

\subsection{Functor Instance}

To show that $\stream{M}$ is a functor whenever $M$ is, we have to define its behaviour on morphisms: if $f:A\rightarrow B$, then we must define how $f$ maps on monadic streams:
$$
\begin{array}{l}
\stream{M}\,f : \stream{M,A} \rightarrow \stream{M,B}\\
\stream{M}\,f\,(\mcons\,m) = \mcons\,(M\,(f\times \stream{M}\,f)\,m)
\end{array}
$$

This definition complies with the {\em guarded-by-constructors} discipline: the recursive call to $(\stream{M}\,f)$ is mapped to the recursive substreams by the functorial application of the functor $M\,(A \times -)$.
That is: $(\stream{M}\,f)$ will be recursively applied only at the recursive positions inside the shape of $m$.
There are also applications of $f$ to the first element (of type $A$) of the pairs in the $M$-position: this is non-recursive, and therefore not problematic.

The Haskell version of the functorial mapping uses a general stream transformer at the top level: \hcode{unwrapMS} maps between \hcode{MonStr m a} and \hcode{MonStr m b} by mapping through $m$ a function on the components:

\begin{haskell}
unwrapMS :: MonStr m a -> m (a, MonStr m a)
unwrapMS (MCons m) = m

transformMS :: Functor m => (a -> MonStr m a -> (b, MonStr m b)) ->
                                  MonStr m a -> MonStr m b
transformMS f s = MCons $ fmap (\(h,t) -> f h t) (unwrapMS s)

instance Functor m => Functor (MonStr m) where
   -- fmap :: (a -> b) -> MonStr m a -> MonStr m b
   fmap f = transformMS (\a s -> (f a, fmap f s))
\end{haskell}

We can now prove that the functor laws are satisfied by $\stream{M}$: it's functorial mapping preserves identities and composition.
The proof are straightforward applications of definitions and the functoriality of $M$ and $\times$, except for the use of coinduction:
We are allowed to invoke the laws themselves in their proofs, as long as we use them only in the direct recursive subterms of the $\mcons$ constructor, that is, in the {\em positions} for the container $M\,(A\times -)$.

\begin{lemma}\label{lemma:functor_id}
The identity functor law holds for monadic streams:
$$
\stream{M}\,\id_A = \id_{\stream{M,A}}
$$
\end{lemma}
\begin{proof}
We apply the left-hand side function to an $M$-monster in constructor form:
$$
\begin{array}{ll}
\stream{M}\,\id_A\,(\mcons\,m)\\
{}= \mcons\,(M\,(\id_A\times \stream{M}\,\id_A)\,m)
  & \mbox{by definition}\\
{}= \mcons\,(M\,(\id_A\times \id_{\stream{M,A}})\,m)
  & \mbox{by coinduction hypothesis}\\
{}= \mcons\,(M\,(\id_{A\times \stream{M,A}})\,m)
  & \mbox{by functoriality of }\times\\
{}= \mcons\,(\id_{M\,(A\times \stream{M,A})}\,m)
  & \mbox{by functoriality of }M\\
{}= \mcons\,m
\end{array}
$$
\end{proof}

\begin{lemma}\label{lemma:functor_comp}
The composition functor law holds for monadic streams:

If $f:A\rightarrow B$ and $g:B\rightarrow C$, then
$$
\stream{M}\,(g\comp f) = (\stream{M}\,g) \comp (\stream{M}\,f)
$$
\end{lemma}
\begin{proof}
Let's again apply the left-hand side function to an $M$-monster in constructor form:
$$
\begin{array}{ll}
\stream{M}\,(g\comp f)\,(\mcons\,m)\\
{}= \mcons\,(M\,((g\comp f)\times \stream{M}\,(g\comp f))\,m)
  & \mbox{by definition of }\stream{M}\mbox{ mapping}\\
{}= \mcons\,(M\,((g\comp f)\times (\stream{M}\,g) \comp (\stream{M}\,f))\,m)
  & \mbox{by coinduction hypothesis}\\ 
{}= \mcons\,(M\,((g\times \stream{M}\,g) \comp (f\times \stream{M}\,f))\,m)
  & \mbox{by functoriality of }\times\\ 
{}= \mcons\,((M\,(g\times \stream{M}\,g) \comp M\,(f\times \stream{M}\,f))\,m)
  & \mbox{by functoriality of }M\\ 
{}= \mcons\,(M\,(g\times \stream{M}\,g)\, (M\,(f\times \stream{M}\,f)\,m))
  & \mbox{by definition of composition}\\ 
{}= \stream{M}\,g\, (\mcons\,(M\,(f\times \stream{M}\,f)\,m))
  & \mbox{by definition of }\stream{M}\mbox{ mapping}\\ 
{}= \stream{M}\,g\, (\stream{M}\,f\,m)
  & \mbox{by definition of }\stream{M}\mbox{ mapping}\\ 
{}= ((\stream{M}\,g) \comp (\stream{M}\,f))\,m
  & \mbox{by definition of composition}
\end{array}
$$
\end{proof}

We can sum up these results by stating that the monster operator is a functor if the underlying ``monad'' is (remember that we are not actually assuming that $M$ is a monad yet, but just a type operator).

\begin{theorem}
If $M$ is a functor, $\stream{M}$ is also a functor.
\end{theorem}

\subsection{Applicative instance in Haskell}

\begin{haskell}
unwrapMS :: MonStr m a -> m (a, MonStr m a)
unwrapMS (MCons m) = m

transformMS :: Functor m => (a -> MonStr m a -> (b, MonStr m b)) ->
                                  MonStr m a -> MonStr m b
transformMS f s = MCons $ fmap (\(h,t) -> f h t) (unwrapMS s)

instance Functor m => Functor (MonStr m) where
   -- fmap :: (a -> b) -> MonStr m a -> MonStr m b
   fmap f = transformMS (\a s -> (f a, fmap f s))
\end{haskell}

\subsection{Comonad instance in Haskell}

\begin{haskell}
instance Comonad w => Comonad (MonStr w) where
  --extract :: MonStr w a -> a
  extract = extract . headMS
  
  --duplicate :: MonStr w a -> MonStr w (MonStr w a)
  duplicate s = MCons $ fmap (\(h,t) -> (s, duplicate t)) (unwrapMS s)
\end{haskell}

\verb+headMS+ returns the first element of the first pair in the monster, wrapped in the underlying functor. Since this functor is required to be a comonad, extract can be used to return the wrapped values inside.

We introduce \verb+mm(s)+ for the "monster matrix" formed by a monadic stream \verb+s+ - this is defined as a monster where the first element is \verb+s+, the second is the tail of \verb+s+, the third is the tail of the tail of \verb+s+, and so on.

The \verb+duplicate+ instance for monadic streams forms this "monster matrix": \verb+duplicate s = mm(s)+

\subsubsection{Comonad law proofs}

Proof that \verb+extract . duplicate == id+

\begin{haskell}
extract . duplicate 
= \ms -> extract (MCons $ fmap (\(h,t) -> (ms, duplicate t)) 
	(unwrapMS ms))
= \ms -> extract (headMS (MCons $ fmap (\(h,t) -> (ms, duplicate t)) 
	(unwrapMS ms)))
= \ms -> ms 
= id
\end{haskell}

Proof that \verb+fmap extract . duplicate == id+

\begin{haskell}
fmap extract . duplicate = \ms -> fmap extract (MCons $ fmap (\(h,t) -> 
	(ms, duplicate t)) (unwrapMS ms))
= \ms -> transformMS (\a s -> (extract a, fmap extract s)) (MCons $ 
	fmap (\(h,t) -> (ms, duplicate t)) (unwrapMS ms))
  [transformMS definition]
= \ms -> MCons $ fmap (\(h,t) -> (extract h, fmap extract t)) 
	(fmap (\(h,t) -> (ms, duplicate t)) (unwrapMS ms)) 
= \ms -> MCons $ fmap (\(h,t) -> (extract ms, fmap extract (duplicate t))) 
	(unwrapMS ms)
  [coinductive hypothesis]
= \ms -> MCons $ fmap (\(h,t) -> ((extract . headMS) ms, id t)) (unwrapMS ms)
  [(extract . headMS) ms = a, first element of ms]
= \ms -> MCons $ fmap (\(h,t) -> (a, id t)) (unwrapMS ms)
  [a is defined as the first element of ms, so h = a under the fmap]
= \ms -> ms
= id
\end{haskell}

Proof that \verb+duplicate . duplicate == fmap duplicate . duplicate+

\begin{haskell}
duplicate . duplicate = \ms -> duplicate (duplicate ms)
= \ms -> MCons $ fmap (\(h,t) -> (duplicate ms, duplicate t)) $
	(fmap (\(h,t) -> (ms, duplicate t)) (unwrapMS ms))
= \ms -> MCons $ fmap ((\(h,t) -> (duplicate ms, duplicate (duplicate t))) 
	(unwrapMS ms)
                      
fmap duplicate . duplicate = \ms -> fmap duplicate (MCons $ fmap (\(h,t) -> 
	(ms, duplicate t)) (unwrapMS ms))
= \ms -> transformMS (\a s -> (duplicate a, fmap duplicate s)) $
	(MCons $ fmap (\(h,t) -> (ms, duplicate t)) (unwrapMS ms))
= \ms -> MCons $ fmap (\(h,t) -> (\a s -> 
	(duplicate a, fmap duplicate s)) h t) (unwrapMS (MCons $ fmap (\(h,t) -> 
	(ms, duplicate t)) (unwrapMS ms)))
= \ms -> MCons $ fmap (\(h,t) -> (duplicate h, fmap duplicate t)) $ 
	(fmap (\(h,t) -> (ms, duplicate t)) (unwrapMS ms))
= \ms -> MCons $ fmap (\(h,t) -> 
	(duplicate ms, fmap duplicate (duplicate t))) (unwrapMS ms)
  [coinductive hypothesis]
= \ms -> MCons $ fmap ((\(h,t) -> 
	(duplicate ms, duplicate (duplicate t))) (unwrapMS ms)
= duplicate . duplicate
\end{haskell}


%\section{Monad proof}
%
\begin{definition}\label{def:monad}
For an endofunctor $T$ on a category $C$, a {\em monad} is a triple $(T, \mu, \eta)$, of $T$ and two natural transformations $\mu : TT \Rightarrow T$ and $\eta : 1_C \Rightarrow T$ ($1_C$ being the identity functor on $C$) such that these diagrams commute:
$$
\setlength\arraycolsep{30pt}
\begin{array}{cc} \ \\
\Rnode{t1}{TA} & \Rnode{tt}{TTA} \\[30pt]
 & \Rnode{t2}{TA} \\ \ 
\end{array}
\psset{nodesep=5pt,arrows=->}
\ncline{t1}{tt} \taput{\eta_{TA}}
\ncline{tt}{t2} \trput{\mu}
\ncline{t1}{t2} \tbput{id_{TA}}
\qquad 
\setlength\arraycolsep{30pt}
\begin{array}{cc} \ \\
\Rnode{t1}{TA} & \Rnode{tt}{TTA} \\[30pt]
 & \Rnode{t2}{TA} \\ \ 
\end{array}
\psset{nodesep=5pt,arrows=->}
\ncline{t1}{tt} \taput{T\eta_{A}}
\ncline{tt}{t2} \trput{\mu}
\ncline{t1}{t2} \tbput{id_{TA}}
$$
$$
\setlength\arraycolsep{30pt}
\begin{array}{cc} \ \\
\Rnode{ttt}{TTTA} & \Rnode{tt1}{TTA} \\[30pt]
\Rnode{tt2}{TTA} & \Rnode{t}{TA} \\ \ 
\end{array}
\psset{nodesep=5pt,arrows=->}
\ncline{ttt}{tt1} \taput{\mu T}
\ncline{ttt}{tt2} \tlput{T\mu}
\ncline{tt2}{t} \tbput{\mu}
\ncline{tt1}{t} \trput{\mu}
$$
\end{definition}

For the functor $\stream{M}$ to be a monad, the monad laws (these commuting diagrams) need to be satisfied. However, assuming a monad $(\stream{M}, \mu, \eta)$, the left identity law (top-left diagram) does not hold in general, given an arbitrary underlying monad $M$. This is due to a correct $\mu$ not being constructible.\\

Here, subscripts of $\eta$ (and $\mu$) indicate their monad, and not components of the natural transformation - these are instead indicated using function application: $\eta_M(x)$ where $x : A$ implies that we are using the component of $\eta_M$ at $A$. \\

First, note that $\eta_{\stream{M}}$ has to be defined as:

$$\eta_{\stream{M}}(x) = mcons_M(\eta_M ((x, \eta_{\stream{M}}(x)))$$ 

There is one choice of $A$ for the first element of the pair, and only one way of lifting pairs into the inner monad $M$, which is by using $\eta_M$. The second element of the pair has to be a $\stream{M,A}$, and the only way to produce this here is by a recursive call.\\

We introduce the notation $\mathop{|}_M^i$ to denote a monadic stream constructor guarded by a monadic action $m_i : A \to MA$. Since each monadic action in $M$ may be different, $i$ is just a label to indicate when they are the same. That is, for any $i$, and a fixed $a : A$:
\begin{align*}
&s_0 : \stream{M,A} = \mathop{|}_M^i a, s_0\\
&s_1 : \stream{M,A} = \mathop{|}_M^i a, s_1\\
&s_0 = s_1
\end{align*}
If $i = \eta$, this monadic action is the monad $M$'s unit, the natural transformation $\eta_M$. For any other label $i$, the monadic action is arbitrary. If there is no label, whether the monadic actions are equal is not being considered.\\

\begin{proof}
If $(\stream{M}, \mu, \eta)$ is a monad, then it needs to obey the left identity law, which in this case is
$$
\mu_{\stream{M}}(\eta_{\stream{M}}(s)) = s
\qquad s : \stream{M,A}
$$
For an arbitrary steam $s$, defined as:
$$
s = \mathop{|}_M^0 a_0, \mathop{|}_M^1 a_1, \mathop{|}_M^2  a_2, \dots
$$
First we construct the $M$-monster of $M$-monsters (termed an $M$-matrix) as per this requirement:
\begin{align*}
&\eta_{\stream{M}}(s) : \stream{M,\stream{M,A}} = \mathop{|}_M^\eta s, \mathop{|}_M^\eta s, \mathop{|}_M^\eta s, \dots
\end{align*}
To take the diagonal, you have to take the first element of the first stream, the second of the second stream, and so on. The law says that $\eta_{\stream{M}}(s)$ needs to be somehow manipulated to form $s$. \\

We reason that $\mu_{\stream{M}}$ cannot be constructed by noting that there is no function out of a monad in general (no function $f : MA \to A$ for an arbitrary functor M over which a monad is defined), and then traversing the $M$-matrix to see which monadic actions \emph{necessarily} have to guard each element in the $M$-monster resulting from an application of $\mu_{\stream{M}}$ to $\eta_{\stream{M}}(s)$.\\

Observe that the first element of the first stream $a_0$ in $\eta_{\stream{M,A}}(s)$ is guarded by two monadic actions, one of them $\eta$, the other an arbitrary action, which we'll call $m_0 : MA$. By the left identity on the monad $M$, $\mu(\eta(m_0)) = m_0$, which indicates that in the absolute best case, the first element of the $M$-monster is guarded by $m_0$.
$$
\eta_{\stream{M}}(s) : \stream{M,\stream{M,A}} = \mathop{|}_M^\eta (\mathop{|}_M^0 \stackrel{\downarrow}{a_0}, \mathop{|}_M^1 a_1, \dots), \mathop{|}_M^\eta s, \dots
$$
Here, "best case" refers to the minimum sequence of monadic actions you have to evaluate to get to a certain element in the stream, removing $\eta$ actions where possible as they act as a unit for Kleisli composition.

Now that we know the first element of the resulting stream $a_0$ will be, in the best case, guarded by $m_0$, we look at what actions guard the second element $a_1$. WLOG, there are two ways to extract $a_1$, either by taking the second element of the second $M$-monster, or by taking the second element of the first $M$-monster. This gives two cases to consider.\\

\paragraph{Case 1:}

In the first case: if we take the second element of the second $M$-monster, then we have to consider what actions guard the element $a_1$ - these in order are $\eta, \eta, m_0, m_1$, obtained by traversing along $\eta_{\stream{M}}(s)$ to the second $s$, and then along $s$ to $a_1$. By using the left identity again, this could reduce in the best case to $m_0(m_1(a_1))$.

$$
\eta_{\stream{M}}(s) : \stream{M,\stream{M,A}} = \mathop{|}_M^\eta s, \mathop{|}_M^\eta (\mathop{|}_M^0 a_0, \mathop{|}_M^1 \stackrel{\downarrow}{a_1}, \dots), \dots
$$

Now realise that in $s$, the element $a_0$ comes before $a_1$. This means that any monadic actions guarding $a_0$, have to also guard $a_1$. Crucially, the $m_0$ actions that guard them \emph{are not the same}, one comes from the first $s$, and the other from the second $s$. This means $a_1$ has to be guarded by \emph{two copies} of $m_0$ - one from the $s$ in the first index of $\eta_{\stream{M}}(s)$ (which is required to guard $a_0$), and one from the second. 

As you cannot get rid of this extra monadic action in general, $s$ is not constructible from $\eta_{\stream{M}}(s)$ in this way. The problem with a duplicated monadic action can be demonstrated with the State monad - if you have an action that adds $3$ to an underlying state, duplicating this would produce an overall action of adding $6$.\\


\paragraph{Case 2:} 

The second case to consider was to take the second element from the first stream. To show this doesn't work generally either, we have to instead consider the right monad identity law. This states that:

$$
\mu_{\stream{M}}(\stream{M}\eta_{\stream{M}}(s)) = s
\qquad s : \stream{M,A}
$$

Where $\stream{M}\eta_{\stream{M}}$ lifts $\eta_{\stream{M}}$ to a function on monadic streams, which effectively applies it to each of the elements (this is our \verb+fmap+ definition from earlier). We redefine the $M$-matrix accordingly:
\begin{align*}
&a_i : A\\
&\eta_{\stream{M}}(a_i) : \stream{M,A} = \mathop{|}_M^\eta a_i, \mathop{|}_M^\eta a_i, \mathop{|}_M^\eta  a_i, \dots\\
&\stream{M}\eta_{\stream{M}}(s) : \stream{M,\stream{M,A}} = \mathop{|}_M^0 \eta_{\stream{M}}(a_0), \mathop{|}_M^1 \eta_{\stream{M}}(a_1), \mathop{|}_M^2 \eta_{\stream{M}}(a_2), \dots
\end{align*}

Using similar logic to before, you can show that $a_0$ is guarded by at least $m_0$. However, if you pick the second element from the second stream, you get another $a_0$. The $\mu_{\stream{M,A}}$ operation has to cross over to the next 'sub-monster' $\eta_{\stream{M}}(a_1)$ to get the value $a_1$. This loops us back to the first case, where we show that doing this causes a duplication of monadic actions, considering the left identity. 

\end{proof}

\subsection{Monad Counter-example}

It's natural to ask, given the chosen name for this data structure, whether monsters themselves are monads, at least when the underlying functor is a monad. This is also a natural question because pure streams (with no functor guarding the elements) are monads. However, it turns out not to be the case in general, and we suspect that more constraints need to be placed on the underlying monad for the monster itself to be a monad. Namely, we believe the underlying monad should be idempotent. \\

An example of why monadic streams aren't monads in general uses State-monsters. The one defined here, when run, generates the stream of integers where the first two differ by $n$, the next two differ by $n+1$, and so on:
\begin{haskell}
fromStep :: Int -> FBMachine Int Int
fromStep n = MCons (state (\x -> ((x, fromStep (n+1)), x+n)))
\end{haskell}
\verb+FBMachine (Int,Int)+ is shorthand for \verb+MonStr (State (Int, Int))+, standing for 'feedback machine' (the meaning is explained in the examples).

For monadic streams to be a monad, we have to be able to join, or flatten, a monster of monsters into a single monster, with respect to certain laws - this is done in Haskell using a function \verb+>>=+, pronounced 'bind'. We also need a function \verb+return+ which injects a pure value into a monster.

The only return operation that makes sense is to apply the return of the underlying monad, and then repeatedly nest this inside itself (the same as \verb+pure+ from the Applicative instance):
\begin{haskell}
return :: Monad m => a -> MonStr m a
return a = MCons $ fmap (\a -> (a, return a) (return a)
\end{haskell}
Bear in mind that the rightmost \verb+return a+ is of type \texttt{a $\to$ m a}, and the one nested inside the tuple is of type \texttt{Monad m $\Rightarrow$ a $\to$ MonStr m a}. \\

One of the monad laws any monad needs to obey is the left identity law, which states: 
\begin{haskell}
f :: a -> MonStr m b
a :: a
return a  >>=  f == f a
\end{haskell}
In Haskell this laws is defined in terms of the bind (\verb+>>=+) operation, but is equivalent to left identity law in the Kleisli category of a monad.

Another monad law that bind has to obey is the right identity:
\begin{haskell}
ma :: MonStr m a
ma  >>= return == ma
\end{haskell}

We use the fact that bind is defined in terms of join:
\begin{haskell}
(>>= f) = join . fmap f 
\end{haskell}
The \verb+join+ function in this case is of type \texttt{MonStr m (MonStr m a) $\to$ MonStr m a}, which can be thought of as flattening a monster of monsters down into a single monster. \\

To show that monsters are not a monad, we reason that for any possible definition of join, there are counter-examples where it doesn't satisfy either the left or right identity laws.\\

Running \verb+fromStep 1+ with \verb+1+ produces the stream:
\begin{haskell}
runFBStr (fromStep 1) 1 = 1 <: 2 <: 4 <: 7 <: 11 <: (*...*)
\end{haskell}

This is the stream where, starting at $1$ for example, you add $1$ to get $2$, then add $2$ to get $4$, then add $3$ to get $7$, and so on. The fact that each step in the stream can be thought of as an action of 'adding some number' is pivotal to the upcoming reasoning.\\

\verb+return 1+ produces a monster where every element is $1$, polymorphic in the monad. \verb+return+ is the left and right unit for composition of Kleisli arrows (in any correctly defined monad), so we ignore any actions produced by it as they have no effect. \\

Since \verb+>>=+ is defined in terms of \verb+fmap+ and \verb+join+, we can start by looking just at \verb+fmap fromStep (return 1)+, and come back to joining the monster of monsters later.
\begin{haskell}
fmap fromStep (return 1) = (fromStep 1) <: (fromStep 1) <: (*...*)
\end{haskell}

Now, to join (flatten) this into a single stream, we need to somehow reconstruct \verb+fromStep 1+. There are three ways to do this, WLOG:

\subsection{Case 1}
We cannot take just the head of each stream, because the first action in each is 'adding $1$' - we would end up with a stream where $1$ is added to the previous element to get the next, which wasn't the definition of \verb+fromStep 1+. This could be thought of as taking the 'horizontal'. 

\subsection{Case 2}
The other naive method, which would work in this case, is to just take the first element of the stream, which \emph{is} \verb+fromStep 1+. We call this taking the 'vertical', which doesn't work in general as shown by another example:

Consider the pure stream \verb+from n+, defined as
\begin{haskell}
from :: Monad m => Int -> MonStr m a 
from n = n <: (from (n+1))
\end{haskell}

\verb+from 0+ gives the stream of natural numbers.
\begin{haskell}
from 0 = 0 <: 1 <: 2 <: (*...*)
\end{haskell}

We now look a the monster of monsters defined with this in mind:
\begin{haskell}
fmap return (from 0) = (return 0) <: (return 1) <: (return 2) <: (*...*)
\end{haskell}

Considering the right identity monad law, it is clear in this case that the join operation cannot be defined as taking the first element of the monster, since \verb+return 0+ $\neq$ \verb+from 0+.

\subsection{Case 3}

The only other way, considering the two examples, is to take the diagonal. This is because, in the first example:
\begin{haskell}
fmap fromStep (return 1) = (fromStep 1) <: (fromStep 1) <: (*...*)
\end{haskell}
we need the first action of the first stream (adding $1$), the second action of the second stream (adding $2$), and so on, to recreate  \verb+fromStep 1+.

However, to get the second action of the second stream (an 'adding $2$' action), we also need to traverse the \emph {first} action of the second stream (another 'adding $1$' action), since each element in a monadic stream is guarded by the previous. This would result in the net effect of 'adding $3$', which isn't what we need to reconstruct the second action of \verb+fromStep 1+. \\

Any other, more convoluted way of trying to get the action of 'adding $2$' falls under either, or a combination of, these three cases, which can be seen by reasoning with the examples given. This shows that you cannot make monadic streams a monad, with the assumption that the underlying functor is a monad.

One solution to this is to only allow the underlying monad to be idempotent. Taking the diagonal here would work fine, since with an idempotent monad, any order in which you join actions results in the same action. This is quite a strong constraint however, and it would be interesting if there is a weaker one.

