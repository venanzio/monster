
\section{Instances of Functor, Applicative, Monad}

Show that monadic streams are an instance of these three classes, under some assumptions about the underlying monad.

So far we determined that we think that MonStr satisfies the monad laws if the underlying monad is commutative and idempotent \cite{idempotent_monads}. (Proof?)

What about Applicative?

Give conterexamples where it isn't a monad (using State).

Also examples when it is a monad even if the underlying monad is not commutative and idempotent: with List as underlying monad we obtain Trees, which are a monad. Can this be generalized.

This section outlines the Functor, Applicative, and Comonad instances

\subsection{Functor Instance}

To show that $\stream{M}$ is a functor whenever $M$ is, we have to define its behaviour on morphisms: if $f:A\rightarrow B$, then we must define how $f$ maps on monadic streams:
$$
\begin{array}{l}
\stream{M}\,f : \stream{M,A} \rightarrow \stream{M,B}\\
\stream{M}\,f\,(\mcons\,m) = \mcons\,(M\,(f\times \stream{M}\,f)\,m)
\end{array}
$$

This definition doesn't strictly comply with the {\em guarded-by-constructors} discipline: the recursive call to $(\stream{M}\,f)$ doesn't occur directly as an argument to the constructor $\mcons$, but it is below the $\times$ operator and an application of $M$.
However, this is just the application of the functor $M\,(- \times -)$ which defines the final coalgebras. In other words, this is actually how we determine the positions in the container tree.
\begin{vcomment}
	This is very badly explained: needs more extensive clarification.
\end{vcomment}
















\subsection{Functor instance in Haskell}

\begin{haskell}
unwrapMS :: MonStr m a -> m (a, MonStr m a)
unwrapMS (MCons m) = m

transformMS :: Functor m => (a -> MonStr m a -> (b, MonStr m b)) ->
                                  MonStr m a -> MonStr m b
transformMS f s = MCons $ fmap (\(h,t) -> f h t) (unwrapMS s)

instance Functor m => Functor (MonStr m) where
   -- fmap :: (a -> b) -> MonStr m a -> MonStr m b
   fmap f = transformMS (\a s -> (f a, fmap f s))
\end{haskell}

\subsubsection{Functor proof}

Proof that \hcode{fmap id == id}

\begin{haskell}
fmap id = transformMS (\a s -> id a, fmap id s)
-- definition of `transformMS`
= \s -> MCons $ fmap (\(h,t) -> (\a s -> (id a, fmap id s)) h t) (unwrapMS s)
-- application of `(\a s -> (id a, fmap id s))` to `h t`
= \s -> MCons $ fmap (\(h,t) -> (id h, fmap id t)) (unwrapMS s)
-- take as assumption that `fmap id t == id t` - coinductive hypothesis?
= \s -> MCons $ fmap (\(h,t) -> (id h, id t)) (unwrapMS s)
-- definition of `id`
= \s -> MCons $ fmap (\(h,t) -> (h,t)) (unwrapMS s)
-- definition of `id`
= \s -> MCons $ fmap id (unwrapMS s)
-- `unwrapMS` returns an element in a functor, so `fmap id == id` in this case
= \s -> MCons $ id (unwrapMS s)
-- application of `id`
= \s -> MCons $ (unwrapMS s)
-- MCons and unwrapMS are inverses
= \s -> id s 
= id
\end{haskell}

Proof that \verb+fmap (f . g) == (fmap f) . (fmap g)+

To simplify notation in the proof, we introduce the notation \verb+tr(f)+ for the expression \verb+\(h,t) -> (f h, fmap f t)+.

We're going to use the fact that the underlying operator \verb+m+ is a functor and so satisfies the functor laws, and also that the pairing operator \verb+(,)+ is functorial in both arguments.

By definition of \verb+fmap+ for monsters, we have that
\begin{haskell}
fmap f s = transformMS (\ a s -> (f a, fmap f s)) s
         = MCons $ fmap (\ (h,t) -> (\a s -> (f a, fmap f s) h t)) (unwrapMS s)
         = MCons $ fmap (\ (h,t) -> (f h, fmap f t)) (unwrapMS s)
         = MCons $ fmap tr(f) (unwrapMS s)
\end{haskell}

We also use the fact that the monster constructor \verb+MCons+ and the function \verb+unwrapMS+ that removes it are inverse of each other.

\begin{haskell}
fmap (f . g) = transformMS (\a s -> (f . g) a, fmap (f . g) s)
-- definition of `transformMS`
= \s -> MCons $ fmap (\(h,t) -> (\a s -> ((f . g) a, fmap (f . g) s)) h t) (unwrapMS s)
-- application of `(\a s -> ((f . g) a, fmap (f . g) s))` to `h t`
= \s -> MCons $ fmap (\(h,t) -> ((f . g) h, fmap (f . g) t)) (unwrapMS s)`
-- take as assumption that `fmap (f . g) t == (fmap f . fmap g) t` - coinductive hypothesis?
= \s -> MCons $ fmap (\(h,t) -> ((f . g) h, (fmap f . fmap g) t)) (unwrapMS s)
-- By fuctoriality of pairing in both components
= \s -> MCons $ fmap (tr(f) . tr(g)) (unwrapMS s)
-- By functoriality of the underlying operator m
= \s -> MCons $ (fmap tr(f)) . (fmap tr(g)) (unwrapMS s)
-- MCons and unwrapMS are inverses
= \s -> MCons $ fmap tr(f)) $ unwrapMS $ MCons $ fmap tr(g) (unwrapMS s)
= \s -> MCons $ fmap tr(f)) $ unwrapMS $ fmap g s
= \s -> fmap f $ fmap g s
= fmap f . fmap g
\end{haskell}

\subsection{Applicative instance in Haskell}

\begin{haskell}
unwrapMS :: MonStr m a -> m (a, MonStr m a)
unwrapMS (MCons m) = m

transformMS :: Functor m => (a -> MonStr m a -> (b, MonStr m b)) ->
                                  MonStr m a -> MonStr m b
transformMS f s = MCons $ fmap (\(h,t) -> f h t) (unwrapMS s)

instance Functor m => Functor (MonStr m) where
   -- fmap :: (a -> b) -> MonStr m a -> MonStr m b
   fmap f = transformMS (\a s -> (f a, fmap f s))
\end{haskell}

\subsection{Comonad instance in Haskell}

\begin{haskell}
instance Comonad w => Comonad (MonStr w) where
  --extract :: MonStr w a -> a
  extract = extract . headMS
  
  --duplicate :: MonStr w a -> MonStr w (MonStr w a)
  duplicate s = MCons $ fmap (\(h,t) -> (s, duplicate t)) (unwrapMS s)
\end{haskell}

\verb+headMS+ returns the first element of the first pair in the monster, wrapped in the underlying functor. Since this functor is required to be a comonad, extract can be used to return the wrapped values inside.

We introduce \verb+mm(s)+ for the "monster matrix" formed by a monadic stream \verb+s+ - this is defined as a monster where the first element is \verb+s+, the second is the tail of \verb+s+, the third is the tail of the tail of \verb+s+, and so on.

The \verb+duplicate+ instance for monadic streams forms this "monster matrix": \verb+duplicate s = mm(s)+

\subsubsection{Comonad law proofs}

Proof that \verb+extract . duplicate == id+

\begin{haskell}
extract . duplicate 
= \ms -> extract (MCons $ fmap (\(h,t) -> (ms, duplicate t)) 
	(unwrapMS ms))
= \ms -> extract (headMS (MCons $ fmap (\(h,t) -> (ms, duplicate t)) 
	(unwrapMS ms)))
= \ms -> ms 
= id
\end{haskell}

Proof that \verb+fmap extract . duplicate == id+

\begin{haskell}
fmap extract . duplicate = \ms -> fmap extract (MCons $ fmap (\(h,t) -> 
	(ms, duplicate t)) (unwrapMS ms))
= \ms -> transformMS (\a s -> (extract a, fmap extract s)) (MCons $ 
	fmap (\(h,t) -> (ms, duplicate t)) (unwrapMS ms))
  [transformMS definition]
= \ms -> MCons $ fmap (\(h,t) -> (extract h, fmap extract t)) 
	(fmap (\(h,t) -> (ms, duplicate t)) (unwrapMS ms)) 
= \ms -> MCons $ fmap (\(h,t) -> (extract ms, fmap extract (duplicate t))) 
	(unwrapMS ms)
  [coinductive hypothesis]
= \ms -> MCons $ fmap (\(h,t) -> ((extract . headMS) ms, id t)) (unwrapMS ms)
  [(extract . headMS) ms = a, first element of ms]
= \ms -> MCons $ fmap (\(h,t) -> (a, id t)) (unwrapMS ms)
  [a is defined as the first element of ms, so h = a under the fmap]
= \ms -> ms
= id
\end{haskell}

Proof that \verb+duplicate . duplicate == fmap duplicate . duplicate+

\begin{haskell}
duplicate . duplicate = \ms -> duplicate (duplicate ms)
= \ms -> MCons $ fmap (\(h,t) -> (duplicate ms, duplicate t)) $
	(fmap (\(h,t) -> (ms, duplicate t)) (unwrapMS ms))
= \ms -> MCons $ fmap ((\(h,t) -> (duplicate ms, duplicate (duplicate t))) 
	(unwrapMS ms)
                      
fmap duplicate . duplicate = \ms -> fmap duplicate (MCons $ fmap (\(h,t) -> 
	(ms, duplicate t)) (unwrapMS ms))
= \ms -> transformMS (\a s -> (duplicate a, fmap duplicate s)) $
	(MCons $ fmap (\(h,t) -> (ms, duplicate t)) (unwrapMS ms))
= \ms -> MCons $ fmap (\(h,t) -> (\a s -> 
	(duplicate a, fmap duplicate s)) h t) (unwrapMS (MCons $ fmap (\(h,t) -> 
	(ms, duplicate t)) (unwrapMS ms)))
= \ms -> MCons $ fmap (\(h,t) -> (duplicate h, fmap duplicate t)) $ 
	(fmap (\(h,t) -> (ms, duplicate t)) (unwrapMS ms))
= \ms -> MCons $ fmap (\(h,t) -> 
	(duplicate ms, fmap duplicate (duplicate t))) (unwrapMS ms)
  [coinductive hypothesis]
= \ms -> MCons $ fmap ((\(h,t) -> 
	(duplicate ms, duplicate (duplicate t))) (unwrapMS ms)
= duplicate . duplicate
\end{haskell}


%\section{Monad proof}
%
\begin{definition}\label{def:monad}
For an endofunctor $T$ on a category $C$, a {\em monad} is a triple $(T, \mu, \eta)$, of $T$ and two natural transformations $\mu : TT \Rightarrow T$ and $\eta : 1_C \Rightarrow T$ ($1_C$ being the identity functor on $C$) such that these diagrams commute:
$$
\setlength\arraycolsep{30pt}
\begin{array}{cc} \ \\
\Rnode{t1}{TA} & \Rnode{tt}{TTA} \\[30pt]
 & \Rnode{t2}{TA} \\ \ 
\end{array}
\psset{nodesep=5pt,arrows=->}
\ncline{t1}{tt} \taput{\eta_{TA}}
\ncline{tt}{t2} \trput{\mu}
\ncline{t1}{t2} \tbput{id_{TA}}
\qquad 
\setlength\arraycolsep{30pt}
\begin{array}{cc} \ \\
\Rnode{t1}{TA} & \Rnode{tt}{TTA} \\[30pt]
 & \Rnode{t2}{TA} \\ \ 
\end{array}
\psset{nodesep=5pt,arrows=->}
\ncline{t1}{tt} \taput{T\eta_{A}}
\ncline{tt}{t2} \trput{\mu}
\ncline{t1}{t2} \tbput{id_{TA}}
$$
$$
\setlength\arraycolsep{30pt}
\begin{array}{cc} \ \\
\Rnode{ttt}{TTTA} & \Rnode{tt1}{TTA} \\[30pt]
\Rnode{tt2}{TTA} & \Rnode{t}{TA} \\ \ 
\end{array}
\psset{nodesep=5pt,arrows=->}
\ncline{ttt}{tt1} \taput{\mu T}
\ncline{ttt}{tt2} \tlput{T\mu}
\ncline{tt2}{t} \tbput{\mu}
\ncline{tt1}{t} \trput{\mu}
$$
\end{definition}

For the functor $\stream{M}$ to be a monad, the monad laws (these commuting diagrams) need to be satisfied. However, assuming a monad $(\stream{M}, \mu, \eta)$, the left identity law (top-left diagram) does not hold in general, given an arbitrary underlying monad $M$. This is due to a correct $\mu$ not being constructible.\\

Here, subscripts of $\eta$ (and $\mu$) indicate their monad, and not components of the natural transformation - these are instead indicated using function application: $\eta_M(x)$ where $x : A$ implies that we are using the component of $\eta_M$ at $A$. \\

First, note that $\eta_{\stream{M}}$ has to be defined as:

$$\eta_{\stream{M}}(x) = mcons_M(\eta_M ((x, \eta_{\stream{M}}(x)))$$ 

There is one choice of $A$ for the first element of the pair, and only one way of lifting pairs into the inner monad $M$, which is by using $\eta_M$. The second element of the pair has to be a $\stream{M,A}$, and the only way to produce this here is by a recursive call.\\

We introduce the notation $\mathop{|}_M^i$ to denote a monadic stream constructor guarded by a monadic action $m_i : A \to MA$. Since each monadic action in $M$ may be different, $i$ is just a label to indicate when they are the same. That is, for any $i$, and a fixed $a : A$:
\begin{align*}
&s_0 : \stream{M,A} = \mathop{|}_M^i a, s_0\\
&s_1 : \stream{M,A} = \mathop{|}_M^i a, s_1\\
&s_0 = s_1
\end{align*}
If $i = \eta$, this monadic action is the monad $M$'s unit, the natural transformation $\eta_M$. For any other label $i$, the monadic action is arbitrary. If there is no label, whether the monadic actions are equal is not being considered.\\

\begin{proof}
If $(\stream{M}, \mu, \eta)$ is a monad, then it needs to obey the left identity law, which in this case is
$$
\mu_{\stream{M}}(\eta_{\stream{M}}(s)) = s
\qquad s : \stream{M,A}
$$
For an arbitrary steam $s$, defined as:
$$
s = \mathop{|}_M^0 a_0, \mathop{|}_M^1 a_1, \mathop{|}_M^2  a_2, \dots
$$
First we construct the $M$-monster of $M$-monsters (termed an $M$-matrix) as per this requirement:
\begin{align*}
&\eta_{\stream{M}}(s) : \stream{M,\stream{M,A}} = \mathop{|}_M^\eta s, \mathop{|}_M^\eta s, \mathop{|}_M^\eta s, \dots
\end{align*}
To take the diagonal, you have to take the first element of the first stream, the second of the second stream, and so on. The law says that $\eta_{\stream{M}}(s)$ needs to be somehow manipulated to form $s$. \\

We reason that $\mu_{\stream{M}}$ cannot be constructed by noting that there is no function out of a monad in general (no function $f : MA \to A$ for an arbitrary functor M over which a monad is defined), and then traversing the $M$-matrix to see which monadic actions \emph{necessarily} have to guard each element in the $M$-monster resulting from an application of $\mu_{\stream{M}}$ to $\eta_{\stream{M}}(s)$.\\

Observe that the first element of the first stream $a_0$ in $\eta_{\stream{M,A}}(s)$ is guarded by two monadic actions, one of them $\eta$, the other an arbitrary action, which we'll call $m_0 : MA$. By the left identity on the monad $M$, $\mu(\eta(m_0)) = m_0$, which indicates that in the absolute best case, the first element of the $M$-monster is guarded by $m_0$.
$$
\eta_{\stream{M}}(s) : \stream{M,\stream{M,A}} = \mathop{|}_M^\eta (\mathop{|}_M^0 \stackrel{\downarrow}{a_0}, \mathop{|}_M^1 a_1, \dots), \mathop{|}_M^\eta s, \dots
$$
Here, "best case" refers to the minimum sequence of monadic actions you have to evaluate to get to a certain element in the stream, removing $\eta$ actions where possible as they act as a unit for Kleisli composition.

Now that we know the first element of the resulting stream $a_0$ will be, in the best case, guarded by $m_0$, we look at what actions guard the second element $a_1$. WLOG, there are two ways to extract $a_1$, either by taking the second element of the second $M$-monster, or by taking the second element of the first $M$-monster. This gives two cases to consider.\\

\paragraph{Case 1:}

In the first case: if we take the second element of the second $M$-monster, then we have to consider what actions guard the element $a_1$ - these in order are $\eta, \eta, m_0, m_1$, obtained by traversing along $\eta_{\stream{M}}(s)$ to the second $s$, and then along $s$ to $a_1$. By using the left identity again, this could reduce in the best case to $m_0(m_1(a_1))$.

$$
\eta_{\stream{M}}(s) : \stream{M,\stream{M,A}} = \mathop{|}_M^\eta s, \mathop{|}_M^\eta (\mathop{|}_M^0 a_0, \mathop{|}_M^1 \stackrel{\downarrow}{a_1}, \dots), \dots
$$

Now realise that in $s$, the element $a_0$ comes before $a_1$. This means that any monadic actions guarding $a_0$, have to also guard $a_1$. Crucially, the $m_0$ actions that guard them \emph{are not the same}, one comes from the first $s$, and the other from the second $s$. This means $a_1$ has to be guarded by \emph{two copies} of $m_0$ - one from the $s$ in the first index of $\eta_{\stream{M}}(s)$ (which is required to guard $a_0$), and one from the second. 

As you cannot get rid of this extra monadic action in general, $s$ is not constructible from $\eta_{\stream{M}}(s)$ in this way. The problem with a duplicated monadic action can be demonstrated with the State monad - if you have an action that adds $3$ to an underlying state, duplicating this would produce an overall action of adding $6$.\\


\paragraph{Case 2:} 

The second case to consider was to take the second element from the first stream. To show this doesn't work generally either, we have to instead consider the right monad identity law. This states that:

$$
\mu_{\stream{M}}(\stream{M}\eta_{\stream{M}}(s)) = s
\qquad s : \stream{M,A}
$$

Where $\stream{M}\eta_{\stream{M}}$ lifts $\eta_{\stream{M}}$ to a function on monadic streams, which effectively applies it to each of the elements (this is our \verb+fmap+ definition from earlier). We redefine the $M$-matrix accordingly:
\begin{align*}
&a_i : A\\
&\eta_{\stream{M}}(a_i) : \stream{M,A} = \mathop{|}_M^\eta a_i, \mathop{|}_M^\eta a_i, \mathop{|}_M^\eta  a_i, \dots\\
&\stream{M}\eta_{\stream{M}}(s) : \stream{M,\stream{M,A}} = \mathop{|}_M^0 \eta_{\stream{M}}(a_0), \mathop{|}_M^1 \eta_{\stream{M}}(a_1), \mathop{|}_M^2 \eta_{\stream{M}}(a_2), \dots
\end{align*}

Using similar logic to before, you can show that $a_0$ is guarded by at least $m_0$. However, if you pick the second element from the second stream, you get another $a_0$. The $\mu_{\stream{M,A}}$ operation has to cross over to the next 'sub-monster' $\eta_{\stream{M}}(a_1)$ to get the value $a_1$. This loops us back to the first case, where we show that doing this causes a duplication of monadic actions, considering the left identity. 

\end{proof}
